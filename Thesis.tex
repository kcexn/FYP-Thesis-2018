\documentclass[a4paper]{report}
\title{Report}
\author{Kevin Exton}
\date{\today}
\newcommand{\tabitem}{~~\llap{\textbullet}~~}
\usepackage{amsmath}
\usepackage[section]{placeins}
\usepackage{cite}
\usepackage[parfill]{parskip}
\usepackage{caption}
\usepackage{subcaption}
\usepackage{graphicx}
\usepackage{cleveref}
\usepackage{amssymb}
\usepackage{url}
\usepackage{mathrsfs}
\usepackage{courier}
\usepackage{rotating}
\usepackage[toc,page]{appendix}
\usepackage[margin=1.1in]{geometry}
\begin{document}
\maketitle
\renewcommand{\abstractname}{Executive Summary}
\begin{abstract}
Real wireless channels we may wish to communicate over %
are often frequency selective and time varying. The rate %
at which the channel varies and the extent to which it %
is frequency selective may greatly limit valid choices of %
signal bandwidth. Orthogonal Frequency Division Multiplexing %
(OFDM) has become a popular modulation scheme in wireless %
communication in large part due to the fact that the overall %
bandwidth of the OFDM symbol and the bandwidth of the %
subcarriers within the OFDM symbol can be decoupled overcoming %
challenges related to frequency selectivity of the wireless channel. %
The time varying nature of the wireless channel still poses a significant %
challenge to successful communication however. This study %
looks at two adaptive equalisation schemes that can track %
the changes in the channel state over time in the frequency %
domain to work with OFDM, the least mean square (LMS) %
filter and the normalised least mean square filter (NLMS). As the %
coherence time of the channel reduces the faster the channel %
changes with time and the more difficult it is to track with %
adaptive equalisation, as such, this study constrains itself to %
only slowly time varying scenarios where adaptive equalisation %
will track the changes in the channel state. A hardware implementation %
has been developed on the National Instruments universal software %
radio platform (USRP) to demonstrate the proof of concept.
\end{abstract}
\tableofcontents
\chapter{Introduction}
\label{ch:Introduction}
Wireless communications systems suffer from a wide %
range of impairments that make the communications %
channel a hostile environment to communicate over. %
The convenience of wireless communication over %
wired communication drives demand for being able %
to accomodate for the wireless environment. %
As the internet of things becomes more ubiquitous %
it will be necessary to develop methods of %
equalisation and filtering that are well suited to the %
wireless channel and are robust enough to function %
on cheaper hardware. Here, the motivating %
scenario for the study done throughout this %
report will be developed as well as a surface level %
coverage of the three major theoretical components, %
orthogonal frequency division multiplexing (OFDM) %
Modulation, wireless channel modeling, and %
adaptive filtering. Chapters \ref{chap:OFDM}, %
\ref{chap:ChannelModeling}, and %
\ref{chap:AdaptiveFiltering} will give a more detailed %
coverage of the theory for OFDM, wireless channel %
modeling and adaptive filtering respectively.

\section{Motivating Scenario}
\label{sec:MotivatingScenario}
The next generation of wireless communication %
dubbed 5G makes promises of a densely connected %
network with numerous devices coming online % TODO: cite this
connected by the radio network \cite{Demestichas13,Hossain15}. %
With this kind of communication network in mind, it is easy %
to imagine a scenario in which a cell needs %
to serve many near static users a steady stream %
of data over a long period of time.% 
\begin{figure}[h!]
	\includegraphics[width=\linewidth]{./Figures/%
	Application_Scenario.png}
	\caption{Application Scenario}
	\label{fig:AppScene}
\end{figure}
Figure \ref{fig:AppScene}. visualises the motivating %
scenario in mind of one base station serving many %
terminals over a bidirectional link. I'll be constraining %
the study done within this report to simply the down-link % 
between the base station and a terminal. This link will %
have a few properties of interest to us. The first is that %
the terminal will be nearly static with very little movement, %
the second, is that the environment is prone to change. We can %
imagine this scenario as being a user, sitting down in a busy %
courtyard watching streaming video on their mobile phone. The %
user equipment is unlikely to move much however the environment %
connecting the user to the base station is prone to change as %
people and vehicles move about. I will model this environment %
as one that will introduce slow time-variations to the wireless %
channel connecting the station to the user. Although this %
particular example is an oversimiplification of how streaming %
video works over standard communication protocols, the idea %
of internet of things (IoT) devices needing to operate in %
slowly time varying conditions efficiently is not too far %
of a stretch from this simple example.

\section{Orthogonal Frequency Division Multiplexing %
(OFDM)}

The simplest way to modulate signals over the wireless %
channel is to use some form of single carrier modulation. %
Single carrier modulation simply takes input data, separates %
it into in-phase and quadrature components if necessary, pulse %
shapes the signal, up converts to the carrier frequency and %
transmits over the wireless channel.
%TODO: add figure to depict single carrier methodology

This method of modulation has some drawbacks. %TODO: find some 
% references that list some drawbacks of single carrier modulation.
The main drawbacks of interest to us will be the complexity of %
channel equalisation. The wireless propagation medium can be %
modeled as a finite impulse response filter %TODO: cite this
and so as the filter length increases estimating, %
the channel becomes an increasingly more complex task. %
A modulation technique that helps to overcome this challenge %
is orthogonal frequency division multiplexing or OFDM. %
OFDM works differently to traditional single carrier %
modulation in the sense that it separates the modulation %
bandwidth into $N$ subcarriers each with a subband bandwidth of %
$\frac{B_total}{N}$. A key advantage of these subcarriers in %
relation to the scenario in \ref{sec:MotivatingScenario} is that %
the the subbands of the OFDM subcarriers can be designed to %
be narrower than the coherence bandwidth of the time %
varying fading channel. More on coherence bandwidth will %
be covered in chapter \ref{chap:ChannelModeling}

%TODO: add figure depicting the serial to parallel OFDM

A cyclic prefix can be added to the beginning of each OFDM %
symbol which gives it the nice mathematical properties %
which make equalisation simple regardless of the length %
of the channel filter. In addition, the parallel nature %
of the signal means that signal processing can happen %
on each subband independently and can reduce the bandwidth %
requirements for signal processing electronics.

The details of OFDM will be developed in Chapter 2. 
%TODO: fix chapter numbers
Where the mathematical construction of OFDM will be
explored as well as the nature of the cyclic prefix.

\section{Channel Modeling}

The wireless communication channel is a %
hostile environment. There are three %
major types of impairments that affect %
wireless communication

\begin{itemize}
	\item{Path loss}
	\item{Thermal electronic noise}
	\item{Fading}
\end{itemize}

Path loss is an impairment that affects %
all forms of electronic communication and is %
caused by the natural decay in amplitude %
that electromagnetic waves undergo as they %
propagate through free space. Path loss %
is proportional to $\frac{1}{d^2}$ where %
$d$ is the distance between the transmitter %
and receiver.

Thermal electronic noise is the noise %
caused by the electronics at the %
transmitter and receiver. The measure %
of thermal noise that is important %
will be the ratio between average %
signal power and average noise power %
commonly referred to as signal-to-%
noise ratio (SNR). The noise is %
commonly fixed by the temperature %
of operation for the electronics. %
The noise power spectral density %
is defined as %

\begin{align}
	N_{0} = \kappa T
\end{align}

where $\kappa$ is the boltzmann %
constant and $T$ is the temperature %
of operation in Kelvin. Noise %
power is typically relative constant %
in a given system and is defined as

\begin{align}
	\sigma^{2} = BT
\end{align}

where $B$ is the bandwidth of the %
signal of interest.

Fading is a stochastic impairment %
that is unique to wireless %
channels and is caused by obstructions %
in the surrounding environment, %
relative movement between the %
transmitter and receiver %
or movement of objects between %
the transmitter and receiver. The %
combination of movement and %
obstructions cause the wave to %
reflect and scatter throughout %
throughout the environment of %
interest before arriving %
at the receiver. This is a %
severe impairment to the wireless %
channel and is central to my %
analysis of adaptive filtering. So %
Chapter \ref{chap:ChannelModeling} %
will develop this in much more detail.

\section{Adaptive Filtering}

In the time varying scenario described earlier in %
section \ref{sec:MotivatingScenario} an single %
static estimate of the wireless channel impairments %
will not give the best equalisation performance as %
the channel will change. Adaptive filters offer a %
a way to track the changes in the wireless channel %
over time so that the optimal solution can be %
followed providing the best equalisation performance %
and minimising the number of bit errors in the received %
bit stream. The filters studied in this report will be %
the least mean square (LMS) filter and the %
normalized least mean square (NLMS) filter. Both %
are considered a member of the class of filters known %
as stochastic gradient filters which minimise mean square %
error by finding the stationary point of the paraboloid %
that defines the mean square error.
\begin{figure}[ht]
	\centering
	\includegraphics[width=0.4\textwidth]{./Figures/%
Circular_Paraboloid_Quadric.png}
	\caption{Circular parabaloid in three dimensions 
	\cite{Paraboloid12}}
\end{figure}

%TODO: Do some reading to find a good way to motivate Adaptive
% filters. Want to compare it to static filters at the 
% very least. Maybe even want to introduce some filter
% theory full stop.

\chapter{Orthogonal Frequency Division Multiplexing (OFDM)}
Chapter 1 introduced the idea of OFDM and why it's %
advantageous to use. This chapter will aim to develop %
OFDM mathematically and provide some theoretical insight %
into how and why it works.
\begin{figure}
	\centering
	\includegraphics[width=\linewidth]{./Figures/Chapter2/%
	OFDM_Transmit.png}
	\caption{OFDM Transmitter block diagram}
	\label{fig:OFDMTransmit}
\end{figure}

\begin{figure}
	\centering
	\includegraphics[width=\linewidth]{./Figures/Chapter2/%
	OFDM_Receive.png}
	\caption{OFDM Receiver block diagram}
	\label{fig:OFDMReceiver}
\end{figure}

Figure \ref{fig:OFDMTransmit} and \ref{fig:OFDMReceiver} %
depict the block diagram of the OFDM transmitter and receiver.

\section{Discrete Fourier Transform and Fast Fourier Transform}

It's long been known that signals can be overlapped in the %
frequency domain with orthogonal frequency spacing %
allowing for efficient utilisation of the %
frequency spectrum\cite{Chang66}.

A key driver to the efficiency and widespread adoption %
of OFDM is the discrete fourier transform (DFT) defined %
as\cite{Rao2010}:

\begin{align}
	X^F(k) = \sum_{n=0}^{N-1}x(n)e^{(\frac{-i 2\pi}{N})kn}, %
	k=0,1,\cdots,N-1,\quad N \text{ DFT coefficients}
\end{align}

And the inverse transform is defined as:

\begin{align}
	x(n) = \frac{1}{N}\sum_{k=0}^{N-1}X^{F}(k)%
	e^{(\frac{i 2\pi}{N})kn}, n = 0,1,\cdots,N-1, \quad%
	N \text{ data samples}
	\label{eq:IDFT}
\end{align}

The discrete fourier transform has the orthogonality %
properties necessary for orthogonal spectral allocation. %
To show this I'll follow the development in \cite{Opp99}. %
multiply both sides of eq \ref{eq:IDFT}. by %
$e^{-j(\frac{2\pi}{N})rn}$ and summing from $n=0$ %
to $n=N-1$ gives:

\begin{align}
	\sum_{n=0}^{N-1}x(n)e^{(\frac{-i 2\pi}{N})rn} = %
	\sum_{k=0}^{N-1}X^{F}(k)\left[\frac{1}{N} \sum_{n=0}^{N-1} %
	e^{(\frac{i 2\pi}{N})(k-r)n}\right]
\end{align}

evaluating the exponential:

\begin{align}
	\frac{1}{N}\sum_{n=0}^{N-1}e^{(\frac{i 2\pi}{N})(k-r)n}%
	= \begin{cases}
		1,\quad k-r = mN, \quad m\text{ an integer},\\
		0,\quad \text{otherwise.}
	\end{cases}
\end{align}

which expresses the orthogonality of the complex exponentials %
$e^{(\frac{i 2\pi}{N})kn}$ and $e^{(\frac{-i 2\pi}{N})rn}$ for %
integer values of $k$ and $r$. This can be interpreted as the %
complex exponentials forming an orthonormal basis demonstrating %
that %
the DFT and IDFT have the orthogonality properties needed for %
OFDM.

By itself the DFT has computational complexity of $\mathcal{O}(%
N^2)$ as there are $N$ multiplication for each coefficient and %
$N$ coefficients in the DFT. Fast versions of the DFT which %
reduce the computational complexity of the DFT to $\mathcal{O}(%
Nlog(N))$ have been fundamental to the adoption of OFDM.%

The fast fourier transform (FFT) used in the remainder of %
this report % TODO: thesis? 
is the default FFT provided by MATLAB built on FFTW\cite{FFTW}. %
\section{Cyclic Prefix}

In addition to the FFT a second principal component to the %
operation of OFDM is the cyclic prefix. The development of %
the cyclic prefix here follows that of \cite{Goldsmith05}.

Consider an input to a wireless channel $x[n] = x[0],x[1],\cdots,%
x[N-1]$ of length $N$ and a discrete-time channel with finite %
impulse response (FIR) $h[n] = h[0],h[1],\cdots,h[\mu-1]$ of length %
$\mu$. A cyclic prefix of $x[n]$ is defined as $\{x[N-(\mu-1)], %
x[N-(\mu-2)], \cdots, x[N-1]\}$ which is appended to the beginning %
of the input, the resulting series of samples $\hat{x} = x[N-(\mu-1)], %
x[N-(\mu-2)], \cdots, x[N-1], x[0], x[1], \cdots, x[N-1]$, figure %
\ref{fig:CyclicPrefix}. visualises $\hat{x}[n]$.

\begin{figure}[h!]
	\includegraphics[width=\linewidth]{./Figures/Chapter2/%
	CyclicPrefix.png}
	\caption{Appending the Cyclic Prefix}
	\label{fig:CyclicPrefix}
\end{figure}

It's clear from the definition of the new sequence $\hat{x}[n]$ that %
$\hat{x}[n]$ is equivalent to $x[n]_{N}$ from $-(\mu-1) \leq n \leq %
N-1$. Where $x[n]_{N}$ denotes the sequence $x[n]$ modulo $N$, that %
is to say the sequence $x[n]$ is periodic in $N$. Given $\hat{x}[n]$ %
as the input to the channel and the channel response begin $h[n]$. %
The channel output is:

\begin{align}
	y[n] =& \hat{x}[n]*h[n]\\
	     =& \sum_{k=0}^{\mu-1}h[k]\hat{x}[n-k]\\
	     =& \sum_{k=0}^{\mu-1}h[k]x[n-k]_{N}\\
	     =& x[n]_{N}\circledast h[n]
\end{align}

where $\circledast$ represents a circular convolution. %

The convolution theorem states that the convolution %
of two  signals is equivalent to point wise %
multiplication in the frequency domain\cite{MathWorldConvolution%
Theorem}. In the special case of the DFT, the convolution theorem %
applies under circular convolution i.e.

\begin{align}
	Y[i] = \text{DFT}\{y[n] = x[n]\circledast h[n]\} = X[i]H[i]	
\end{align}

This powerful result allows for the easy equalisation of the received %
signal $y[n]$ in the frequency domain, as the sent signal $X[i]$ can %
retrieved with a simple division.

\begin{align}
	X[i] = \frac{Y[i]}{H[i]}
\end{align}

\chapter{Wireless Channel Modeling}
\label{chap:ChannelModeling}
\section{Fading Channels}

Fading is a major component of wireless channel %
impairments. Fading comes under two main categories %
large scale and small scale fading. Figure \ref{fig:Fading} %
illustrates these two main components of fading.
\FloatBarrier
\begin{figure}[ht]
	\centering
	\includegraphics[width=0.7\textwidth]
		{./Figures/%
		WirelessChannel/LargeandSmallScale%	
		Fading.png}
	\caption{combination of large and small %
		scale fading (a) and the small-scale %
		fading component (b) from %
		\cite{Sklar97-1}}
	\label{fig:Fading}
\end{figure}
\FloatBarrier
\subsection{Large Scale Fading}

Large scale fading, also sometimes referred to as %
shadow fading is characterised by an attenuation %
of the average signal power as can be seen in %
figure \ref{fig:Fading}a). It is caused %
by significant obstructions to the signal such as %
buildings or hills between the transmitter and %
receiver. The receiver is therefore being %
\emph{shadowed} by the obstruction. 

The received signal power can be modeled as:
\begin{align}
	S_{r} = S_{t} + G_{t} + G_{r} - L_{p}
\end{align}
Where $S_{r}$ is the received signal power in dB, %
$S_{t}$ is the transmit signal power in dB, $G_{t}$ 
and $G_{r}$ are the transmit and receive antenna %
gains in dB and $L_{p}$ is the propagation loss in dB. %
Typically $S_{r}$, $S_{t}$, $G_{t}$, and $G_{r}$ are %
either well known or are easily modeled, in the case of %
fading channels the propagation loss is the most difficult %
to predict \cite{Jer00}. There are two main methods %
of attempting to model $L_{p}$, ray tracing methods %
which must be location specific and statistical models. %
I will constrain this section to the development of %
statistical models. 

The most popular of the statistical models are the %
class of slope-intercept models \cite{Jer00}. These %
models treat propagation loss as being composed of %
a deterministic component and a statistical component.
\begin{align}
	L_{p} = \alpha + \beta log_{10}(R) + \gamma \text{ dB}
\end{align}
where $R$ is the distance from the transmitter to the %
receiver in kilometres, $\alpha$ and $\beta$ are parameters %
determined by the model, and $\gamma$ is the %
statistical component of the model. Values for $\alpha$ and %
$\beta$ are determined from experimental measurements %
where received signal power averaged over several %
wavelengths are taken over many different locations %
around the transmitter. These received signal %
powers can be plotted against the log of %
the distance to the transmitter and a least squares fit %
can be made to the data. The resulting residuals around %
the least squares fit describes $\gamma$.

Some well-known slope intercept models are the Hata%
\cite{Hata80} model and the COST-231 model\cite{COST231}.

Typically the residuals  of the slope intercept models %
when measured in dB follow a gaussian distribution with %
zero mean and a standard deviation of about 8dB\cite{Jer00}.
So the large scale fading follows a log-normal distribution.

\subsection{Small Scale Fading}

Small scale fading, also commonly referred to as multipath %
fading is caused by the transmitted signal following many %
different paths before arriving at the receiving antenna. %
Each of these multiple paths can have different lengths %
and may be reflected or refracted many times, a process %
typically referred to as scattering. As a result of the %
varying distances each propagation path takes, the time %
taken for a transmitted waveform to reach the receiver %
is also going to vary down each of these paths, figure %
\ref{fig:MultipathChannel} illustrates this multipath %
behaviour.
\begin{figure}[ht]
	\centering
	\includegraphics[width=0.8\textwidth]{./Figures/%
		WirelessChannel/MultipathChannel.png}
	\caption{Multipath Channel \cite{Jer00}}
	\label{fig:MultipathChannel}
\end{figure} 
The resulting effect of this multipath channel is that %
symbols transmitted at different times will arrive %
at the receiver at the same time, this leads to %
a self-interfering effect that can cause the received %
signal power to undergo large fluctuations and can %
be characterised as inter-symbol interference.

The multipath fading can be broadly organised into %
two separate categories\cite{Jer00}.
\begin{enumerate}
	\item{The multipath signal paths are made up %
		of relatively small and identifiable number %
		of components reflected by small hills, %
		houses, and other strucutres encountered in %
		open areas and rural environments. This %
		results in a channel model with a finite number %
		of multipath components. Such a channel is %
		referred to as a \emph{discrete multipath channel}.}
	\item{The multipath signal paths are generated by %
		a large unresolvable reflections as might occur %
		in a mountainous area or in a dense urban %
		environment. %
		This signal is composed of a continuum of %
		unresolvable multipath components. This %
		channel model is referred to as a \emph{diffuse %
		multipath channel}.}
\end{enumerate}
Multipath reflected components can be described in %
terms of orthogonal components $x_{n}(t)$ and %
$y_{n}$%
(t), where $x_{n}(t) + jy_{n}(t) = \alpha_{n}(t)e^{%
-j\theta_{n}(t)}$. If the number of random components %
is large enough and none are dominant, then the received %
components $x_{r}(t)$ and $y_{r}(t)$ which are the sums %
of the reflected signals will have a gaussian probability %
density function. The magnitude of the received scattered %
components will have a magnitude described by:
\begin{align}
	r_{0}(t) = \sqrt{x_{r}^{2}(t) + y_{r}^{2}(t)}
\end{align}
The probability density function of the envelope $r_{0}(t)$ %
of the received signal is going to follow a Rayleigh distribution %
with probability density function:
\begin{align}
	f_{R}(r_{0}) = \begin{cases}
		\frac{r_{0}}{\sigma^{2}}e^{-r_{0}^{2}/(2\sigma^{2})} %
		& \text{ for } r_{0} \geq 0 \\
		0 & \text{ otherwise}
	\end{cases}
	\label{eq:RayleighPDF}
\end{align}
Where $\sigma^{2}$ is the mean power of the received %
multipath signal. This type of fading is called Rayleigh %
fading.
\begin{figure}[h!]
	\centering
	\includegraphics[width=0.8\textwidth]{./Figures/%
	WirelessChannel/RayleighDistribution.png}
	\caption{Gaussian and Rayleigh Distributions%
	\cite{Jakes74}}
	\label{fig:RayleighDistribution}
\end{figure}

Figure \ref{fig:RayleighDistribution} illustrates %
the gaussian distribution and the rayleigh %
distribution.

When the received signal is made up of multiple %
reflected rays and a significant line-of-sight %
component, the received envelope ampliude %
follows a Rician distribution with probability density %
function:
\begin{align}
	f_{R}(r_{0}) = \begin{cases}
		\frac{r_{0}}{\sigma^{2}}I_{0}%
		\left[ \frac{A_{r}}{\sigma^{2}} \right]%
		e^{-(r_{0}^{2}-A^{2})/(2 \sigma^{2})}%
		 & \text{ for } r_{0} \geq 0, A \geq 0\\
		0 & \text{ Otherwise}
	\end{cases}
	\label{eq:RicianPDF}
\end{align}
where $I_{0}$ is the modified zeroth order %
bessel function of the first kind. This kind of %
fading is commonly referred to as Rician fading. %
The Rician distribution is commonly described in %
terms of a parameter K\cite{Sklar01}, which is defined as %
the ratio of power in the specular component to the %
power in the multipath signal and is given as:
\begin{align}
	K = \frac{A^{2}}{2\sigma^{2}}
	\label{eq:KFactor}
\end{align}
Figure \ref{fig:RicianDistribution} illustrates the relationship %
of the rician probability density function to the rayleigh %
PDF with respect to mean specular signal amplitude $v$ and %
mean scattered signal power of $1$.

\begin{figure}[ht]
	\centering
	\includegraphics[width=0.8\textwidth]{./Figures/%
	WirelessChannel/RicianDistribution.png}
	\caption{Rayleigh and Rician Probability %
	Density Function}
	\label{fig:RicianDistribution}
\end{figure}

\section{Time and Frequency Characteristics of Fading %
Channels}
\FloatBarrier

Figure \ref{fig:FadingBlocks} provides a broad %
description of the important components within %
fading channels. In this section we will draw our %
attention to blocks $5$ and $6$ under small %
scale fading.
\begin{figure}[ht]
	\centering
	\includegraphics[width=\textwidth]{./Figures/%
		WirelessChannel/FadingBlockDiagram%
		.png}
	\caption{Fading Channel manifestations%
		\cite{Sklar97-1}}
	\label{fig:FadingBlocks}
\end{figure}
As is apparent from the diagram, small-scale fading %
can be described by two key components, the time %
spreading of the signal or dispersion, and the time %
variance of the channel. 

Bello \cite{Bello63} proposed a simple wide-sense %
stationary uncorrelated scattering model of fading in %
1963. This model treats the received signals arriving %
with different delays as being uncorrelated and that %
the channel is wide sense stationary in both the time %
and frequency domains. Using this model four functions %
can be defined that characterise the fading as shown in %
figure \ref{fig:FadingFunctions}.
\begin{figure}[ht]
	\centering
	\includegraphics[width=0.75\textwidth]{./Figures/%
		WirelessChannel/FadingFunctions.png}
	\caption{Four Functions that define Fading in %
			a WSSUS channel \cite{Sklar01}}
	\label{fig:FadingFunctions}
\end{figure}

\FloatBarrier
\subsection{Time Spreading}
In this section we will focus on figure %
\ref{fig:FadingFunctions}a) and \ref{%
fig:FadingFunctions}b). The multipath %
intensity profile also referred to as the %
power delay profile plots the received %
signal power vs the excess delay. %
Excess delay is defined as the propagation %
delay of the signal that exceeds the delay %
of the first signal arrival at the receiver %
\cite{Sklar01}.

In the fading channel, the maximum excess %
excess delay time $T_{m}$ and symbol time %
$T_{m}$ can be used to categorise the %
fading into two categories:
\begin{itemize}
	\item{frequency-selective fading}
	\item{frequency nonselective or flat fading}
\end{itemize}
The channel is frequency selective when %
$T_{m} > T_{s}$. This occurs when the %
excess delay induced by the multipath %
components of the symbol last longer %
than the symbols time duration. This leads %
to an effect where the multipath components %
of one symbol can interfere with the next symbol. %
This effect is named channel induced ISI. The %
channel exhibits flat fading when %
$T_{m} < T_{s}$. This is when %
the multipath components of a symbol %
all arrive within the symbol time, the multipath %
components are all contained within the symbol and %
do not interfere with neighbouring symbols so ISI %
does not happen, there is however still signal %
degradation since the multipath components still %
self-interfere.

This same time spreading effect can be analysed in %
the frequency domain. If the fourier transform of %
$S(\tau)$ illustrated in figure \ref{fig:FadingFunctions}a) %
is taken then the function $\lvert %
R(\Delta f) \rvert$ as illustrated in figure %
\ref{fig:FadingFunctions}b) is revealed. This function, %
referred to as the \emph{spaced-frequency correlation %
function} represents the correlation between the channel %
response to two signals as a function of frequency %
difference between the two signals. It can be thought of %
as the frequency transfer function of the channel and so %
the time spreading effect can be understood as the result %
of the channel filter.
\FloatBarrier
The spaced-frequency correlation function allows us to %
understand the similarity between two received signal %
that are spaced in frequency $\Delta f = f_{1} - f_{2}$. %
Frequency selective fading can be understood as the %
spaced-frequency correlation function is poorly correlated %
across the signal bandwidth, and flat fading can be understood %
as the spaced-frequency correlation function being highly %
correlated across the signal bandwidth. An important %
parameter, the \emph{coherence bandwidth} $f_{0}$ %
can be used as a statistical measure of the bandwidth %
of which the signal of interest experiences equal gain %
and linear phase. $f_{0}$ and $T_{m}$ are reciprocally %
related and as an approximation
\begin{align}
	f_{0} \approx \frac{1}{T_{m}}
\end{align}
\begin{figure}[ht]
	\includegraphics[width=\textwidth]{./Figures/%
		WirelessChannel/FrequencyCoherence%
		.png}
	\caption{frequency selective and flat fading %
	\cite{Sklar01}}
	\label{fig:FadingFigure}
\end{figure}
Figure \ref{fig:FadingFigure} illustrates the difference %
between frequency selective and flat fading in the %
frequency domain. $W$ is the transmitted signal %
bandwidth and $f_{0}$ is the coherence bandwidth.
\FloatBarrier
The maximum excess delay $T_{m}$ is not always the %
best indicator of channel performance, as channels with %
the same $T_{m}$ can exhibit different power delay profiles. %
A different parameter that is used for understanding %
coherence bandwidth is the root-mean-squared delay spread.
\begin{align}
	\sigma_{\tau} = \sqrt{\overline{\tau^{2}} - (\bar{\tau}^{2})}
\end{align}
where $\overline{\tau^{2}}$ and $\bar{\tau}^{2}$ are defined %
as:
\begin{align}
	\overline{\tau^{2}} =&  
	\frac{\int \tau^{2} S(\tau) d\tau}{\int S(\tau) d\tau} \\
	\bar{\tau} =& \frac{\int \tau S(\tau) d\tau}{\int S(\tau) 
	d\tau}
\end{align}
Using this rms delay a popular approximation of %
the coherence bandwidth where the correlation %
of the signal within this bandwidth is at least 0.5 is
\begin{align}
	f_{0} \approx \frac{1}{5\sigma_{\tau}}
\end{align}
\subsection{Time Variance}
Time variance of the channel is caused by relative %
motion between the transmitter and receiver or by %
objects moving about within the channel \cite{Sklar01}. %
Figure \ref{fig:FadingFunctions}c) shows the function %
$R(\Delta t)$ named the \emph{spaced-time correlation %
function}. Similar to the spaced-frequency correlation %
function $R(\Delta t)$ measures the similarity between %
two signals in the \emph{time} domain. It's measured %
by sending two sinusoids through the channel at differing %
times $\Delta t = t_{1} - t_{2}$ and evaluating their %
cross correlation. 

Just as the spaced-frequency correlation function has a %
measure of coherence bandwidth, the spaced-time %
correlation function has a measure of \emph{coherence %
time} $T_{0}$. The coherence time is a measurement of %
the duration the channel is invariant. 

The coherence time of the channel allows the fading %
to be categorised two ways as in figure \ref{fig:FadingBlocks}, %
blocks 14, 15, 17, 18.
\begin{itemize}
	\item{fast fading}
	\item{slow fading}
\end{itemize}
Fast fading occurs when the coherence time $T_{0}$ %
is less than the symbol time $T_{s}$ ie.
\begin{align}
	T_{0} < T_{s}
\end{align}
Slow fading occurs whne the coherence time is %
greater than the symbol time
\begin{align}
	T_{0} > T_{s}
\end{align}
In fast fading the received signal pulse undergoes %
distortion that leaves it uncorrelated throughout time. %
A slow fading channel does not induce this pulse distortion %
as the channel is relatively invariant throughout one symbol %
period.

It's been shown \cite{Clarke68} that under the popular %
dense scatterer model the normalized $R(\Delta t)$ is
\begin{align}
	R(\Delta t) = J_{0}(kV \Delta t)
\end{align}
where $J_{0}(\cdot)$ is the zeroth order Bessel %
function of the first kind. $V$ is the relative velocity %
between the transmitter and receiver and %
$k = \frac{2\pi}{\lambda}$ is the free-space %
phase constant. An interesting point about %
this expression is that it can be expressed both %
in time or distance (where distance is $V\Delta t$). %
It has been shown \cite{Amoroso96} that at a %
distance of $0.38\lambda$ from the point of %
reference that the combined magnitudes and %
phases of a continuous wave signal are %
statistically uncorrelated.

The frequency domain interpretation of %
this time variation is that of a doppler-shift %
represented in the frequency domain. Figure %
\ref{fig:FadingBlocks}d) shows the \emph{%
Doppler power spectral density} or simply %
Doppler spectrum. For the case of the dense %
scatterer model, a vertical receive antenna with %
constant azimuthal gain, a uniform distribution of %
signals arriving at all arrival angles, and an a %
continuous wave signal the doppler spectrum is %
\cite{Clarke68}
\begin{align}
	S(\nu) = \frac{1}{\pi f_{d} \sqrt{1 - %
		(\frac{\nu - f_{c}}{f_{d}})^{2}}}
	\label{eq:JakesSpectrum}
\end{align}
Where $\nu$ is the frequency shift, $f_{d}$ is the %
doppler frequency and $f_{c}$ is the carrier frequency. %
This spectrum is known as the Jakes' spectrum\cite{Iskander}. %
The doppler shift is the result of the fourier transform %
of the spaced-time correlation function similar to how %
the spaced-frequency correlation function is the %
Fourier transform of the power delay profile. The %
shape of this spectrum is bowl shaped as can be seen in %
figure \ref{fig:FadingFunctions}d).

The magnitude of the doppler shift is given by:
\begin{align}
	f_{d} = \frac{V}{\lambda}
\end{align}
where $V$ is the relative velocity between the %
transmitter and receiver and $\lambda$ is the signal %
wavelength.

\section{Wireless Channel as a Filter}

Multipath fading can be modeled as a kind of %
finite impulse response filter where each %
sample at the receiver is a linear combination %
of the samples transmitted over the multipath %
channel.
\begin{align}
	y(t) = \sum_{n} \alpha_{n}(t)s(t-\tau_{n}(t))
	\label{eq:ChannelFIR}
\end{align}
where $s(t)$ is the bandpass input signal, $\alpha_n(t)$ %
is the attenuation factor for the signal received %
on the $n\text{th}$ path, and $\tau_n(t)$ is the %
propagation delay along that path. A natural model %
for equation \ref{eq:ChannelFIR} is that of a %
tapped-delay line with time-varying coefficients. %
If the bandpass signal $s(t)$ is represented as:
\begin{align}
	s(t) = Re\left\{ \tilde{s}%
	(t)e^{j2\pi f_{c} t} \right\}
\end{align}
where $\tilde{s}$ is the complex baseband signal. %
Then the channel output $y(t)$ can be represented as:
\begin{align}
	y(t) = Re\left\{\left( \sum_{n} %
	\alpha_{n}(t) e^{-j2 \pi f_{c} \tau_{n}(t)} %
	\tilde{s}(t - \tau_{n}(t)) \right)%
	e^{j 2\pi f_{c} t} \right\}
\end{align}
the complex envelope of $y(t)$ is:
\begin{align}
	\tilde{y}(t) =& \sum_{n}\alpha_{n}(t)%
	e^{-j 2\pi f_{c} \tau_{n}(t)} %
	\tilde{s}(t - \tau_{n}(t))
		     =& \sum_{n}%
	\tilde{\alpha}_{n}(t)\tilde{s}(%
	t-\tau_{n}(t))
	\label{eq:ChannelOutput}
\end{align}
Equation \ref{eq:ChannelOutput} makes it clear that %
the channel response can be represented using a %
complex, baseband equivalent impulse response.
\begin{align}
	\tilde{h}(\tau;t) = \sum_{n} %
	\alpha_{n}(t)\delta(\tau - %
	\tau_{n}(t))
\end{align}
where the output is
% I need to check this with the latest textbook
% When I get home.
\begin{align}
	\tilde{y}(t) = \sum_{n} \tilde{h}%
	(\tau;t)\tilde{s}(t - \tau)
	\label{eq:BasebandChannelFIR}
\end{align}
If $h(\tau;t)$ was time-invariant this would be the %
equivalent of a normal convolution sum.


\chapter{Adaptive Filtering}

The previous chapter on wireless channel modeling established %
the wireless channel as a type of FIR filter. The aim of the %
receiver is to undo the effects of the channel filter. %
Figure \ref{fig:CommSysModel}. shows the communications %
system model as described in \cite{Sklar01}, for this section %
we'll draw our attention to the section of the model beginning %
at pulse modulation and ending at detection. 

\begin{figure}[h!]
	\includegraphics[width=\linewidth]{./Figures/Adaptive%
		Filters/CommunicationsSystemModel.png}
	\caption{Communications System Model \cite{Sklar01}}
	\label{fig:CommSysModel}
\end{figure}

The wireless system transfer function can be defined as:

\begin{align}
	H(f) = H_{t}(f)H_{c}(f)H_{r}(f)
\end{align}

where $H_{t}$ is the transfer function of the transmit filter, 
defined in the pulse modulate block, %
$H_{c}(f)$ is the transfer function of the channel, and $H_{r}%
(f)$ is the transfer function of the receive filter, defined in %
the demodulate and sample block. It's clear from this %
expression that the received baseband symbols at the %
detect block will be distorted by the channel filter $H_{c}%
(f)$, the goal of equalisation is to define an equalisation %
transfer function $H_{e}(f)$ that removes the effect of %
the channel filter. Such that the new system transfer %
function looks like this:

\begin{align}
	H(f) = H_{t}(f)H_{c}(f)H_{r}(f)H_{e}(f)
\end{align}

where

\begin{align}
	H_{c}(f)H_{e}(f) = 1
\end{align}



%TODO: insert diagram of the channel filter followed by
% an equalising filter

%TODO: provide a mathematical description of this equalisation
% effect

Throughout this entire chapter my development of adaptive filters %
will quite closely follow \cite{Hay02}. I'll start by %
developing the zero-forcing solution and the minimum mean square %
error solution also known as the Wiener solution. I'll then develop %
the stochastic gradient and the least mean square solution. A brief %
development of the recursive least square will be covered here as %
well.

\section{A Static Filter and The Wiener Solution}
% Need to introduce this section
\subsection{A Simple Filter the Zero-Forcing Solution}
\label{subsec:ZeroForcing}
Previously in chapter \ref{chap:OFDM} on OFDM, a key insight to %
the advantage of using the DFT was that the circular convolution %
in combination with the convolution theorem provided the nice %
property that the distortion introduced by the FIR channel filter %
introduced in chapter \ref{chap:ChannelModeling} can be represented %
as element wise multiplication in the frequency domain (see equation %
\ref{eq:DFTConvolutionTheorem}). It's clear that equation %
\ref{eq:ZeroForcing} does in fact represent a filtering operation %
in the frequency domain. This simple filter is commonly known as %
the zero forcing filter and is defined as the reciprocal of the %
frequency response of the channel filter $\frac{1}{H(f)}$. %
It's most easily found when a known set of symbols $X[i]$ have %
been transmitted such that
\begin{align}
	H\left[i\right] = \frac{Y\left[i\right]}{X\left[i\right]}
\end{align}
In a noiseless and time invariant environment this is the optimal %
solution to the equalisation problem. However a more accurate %
model for the received samples $y[nT]$ is
\begin{align}
	y[nT]=h[nT]\circledast x[nT]%
	+ v
\end{align}
where $h[nT]$ represents the channel filter of length %
$\mu$ at time $nT$ and $x[nT]$ are the inputs to %
$h[nT]$, $v$ represents the additive white gaussian %
noise (AWGN). Taking the DFT of both sides gives %
\begin{align}
	Y[i] = H[i]X[i] + V[i]%
	,\quad \text{ for } i = 0,1,\cdots,N-1
	\label{eq:AWGNModel}
\end{align}
Where $N$ is the length of the DFT. It's clear from %
equation \ref{eq:AWGNModel} that dividing through %
by $X[i]$ gives a noisy estimate of the channel %
frequency response
\begin{align}
	\frac{Y[i]}{X[i]} = H[i] + \frac{V[i]}{X[i]}, %
	\text{ for } i = 0,1,\cdots,N-1
\end{align}
equalising using this expression will lead to estimates %
of $X$
\begin{align}
	\hat{X}[i] =& \frac{H[i]X[i]+V[i]}{H[i]+V[i]/X[i]} \\
	=& \frac{H[i]X[i]}{H[i]+V[i]/X[i]} + \frac{V[i]}{H[i]
	+ V[i]/X[i]} \\
	=& \frac{H[i]}{H[i]+V[i]/X[i]}X[i] + 
	\frac{V[i]X[i]}{H[i]X[i]+V[i]}\\
	=& \frac{H[i]}{H[i]+V[i]/X[i]}X[i] + 
	\frac{V[i]X[i]}{Y[i]}
	\label{eq:NoiseAmplification}
\end{align}
Should $X[i]/Y[i]$ in equation \ref{eq:NoiseAmplification} %
be greater than $1$ then the noise in the received estimate %
$\hat{X}[i]$ will be amplified reducing SNR and increasing the bit %
error rate (BER).

\subsection{The Wiener Solution}
A solution to the noise amplification problem seen in %
section \ref{subsec:ZeroForcing} is to find the optimal %
solution taking noise into consideration. It's clear that %
this problem can be framed as an optimisation problem %
and so a measure of performance or cost function must %
first be defined to optimise. A natural choice for cost function %
is that of the mean square value of the estimation error. This %
is because the mean square error criterion results in a %
second order dependence for the cost function on the %
unknown coefficients in the impulse response of %
the filter. This leads to the cost function having a distinct %
minimum that uniquely defines the optimum %
statistical design of the filter \cite{Hay02}.
\subsubsection{The Wiener-Hopf Equations}
To develop this optimum solution first we revisit equation %
\ref{eq:BasebandChannelFIR} which describes the wireless channel %
as an FIR filter and for consistency with \cite{Hay02} and %
with my later development on my system model redefine it as
\begin{align}
	u(n) = \sum_{i}\tilde{h}(\tau)\tilde{s}(t-\tau)
	\label{eq:ChannelOut}
\end{align}
where $u(n)$ are now the series of inputs to an optimally %
designed filter with form shown in equation \ref{eq:FilterOut}. %
For the purposes of this chapter %
we will assume that the channel is time-invariant and set %
$\tilde{h}(\tau;t)$ to $\tilde{h}(\tau)$.
\begin{align}
	y(n) = \sum_{k=0}^{N-1}w_{k}^{*}u(n-k), n - 0,1,2,\cdots,N-1
	\label{eq:FilterOut}
\end{align}
where $u(n-k)$ are the discrete samples being received %
after having been distorted by the wireless channel, %
$w_{k}^{*}$ are the %
filter coefficients and the asterisk $^*$ represents %
complex conjugation. The estimation error will be defined as 
\begin{align}
	e(n) = d(n) - y(n)
	\label{eq:EstimationError}
\end{align}
where $d(n)$ is the desired output of the filter %
hence the %
mean square error criterion will be defined as
\begin{align}
	J =& E\left[e(n)e^{*}(n)\right]
	\label{eq:MeanSquareError}\\
	=& E\left[\lvert e(n) \rvert^{2}\right]
\end{align}
where $E\left[\cdot\right]$ denotes the expectation %
operator. With these definitions in mind the aim is now %
to minimise $J$. To do this the complex derivative of $J$ with %
respect to the filter coefficients $w_{k}^{*}$ needs %
to be found. First we'll define
\begin{align}
	w_{k}^{*} = a_{k} + jb_{k}
\end{align}
and 
\begin{align}
	\nabla_{k} = \frac{\partial}{\partial a_{k}} + j 
	\frac{\partial}{\partial b_{k}}
\end{align}
where $\nabla_{k}$ is the complex derivative %
operator with respect to $w_{k}^{*}$. From %
equations \ref{eq:FilterOut}, \ref{eq:EstimationError} %
and \ref{eq:MeanSquareError} %
\begin{align}
	\nabla_{k}J =& E\left[ \frac{\partial e(n)}
	{\partial a_{k}} e^{*}(n) + \frac{\partial
	e^{*}(n)}{\partial a_{k}}e(n) + \frac{
	\partial e(n)}{\partial b_{k}}je^{*}(n) 
	+ \frac{\partial e^{*}(n)}{\partial b_{
	k}}je(n)\right] \label{eq:DerivativeJ}
\end{align}
When evaluated 
\begin{align}
	\nabla_{k}J = -2E\left[u(n-k)e^{*}(n)\right]
\end{align}
Since the mean square error criterion is quadratic %
the optimum solution can be found at its stationary %
point by setting
\begin{align}
	\nabla_{k}J =& 0,\quad\text{ for } k=0,1,2,\cdots,N-1
\end{align}
which implies that
\begin{align}
	E\left[u(n-k)e^{*}(n)\right] = 0,\quad\text{ for } k = 0,1,2,\cdots,N-1
	\label{eq:OptimalityCondition}
\end{align}
substituting equation \ref{eq:EstimationError} into %
\ref{eq:OptimalityCondition} gives
\begin{align}
	E\left[u(n-k)\left(d^{*}(n) - \sum_{m=0}^{N-1}w_{o_{m}}u
	^{*}(n-m)\right)\right]=0,\quad\text{ for }k=0,1,\cdots,N-1
\end{align}
where $w_{o_{m}}$ is the $m\text{th}$ coefficient of %
the impulse response of the optimum filter. Expanding and %
rearranging
\begin{align}
	&E\left[u(n-k)d^{*}(n) - u(n-k)\sum_{m=0}^{N-1}w
	_{o_{m}}u^{*}(n-m)\right] = 0\\
	\implies &E\left[u(n-k)d^{*}(n)\right] - E\left[u(n-k)\sum
	_{m=0}^{N-1}w_{o_{m}}u^{*}(n-m)\right] = 0\\
	\implies &E\left[u(n-k)d^{*}(n)\right] - \sum_{m=0}^{N-1}
	w_{o_{m}}E\left[u(n-k)u^{*}(n-m)\right] = 0\\
	\implies &E\left[u(n-k)d^{*}(n)\right] = 
	\sum_{m=0}^{N-1}w_{o_{m}}E\left[u(n-k)u^{*}(
	n-m)\right],\quad\text{for }k=0,1,\cdots,N-1
	\label{eq:ExpandedOptimalityCondition}
\end{align}
The expectation on the right side of equation %
\ref{eq:ExpandedOptimalityCondition} is the %
\emph{autocorrelation function} of the filter input %
for a lag of $m-k$ and can be expressed as
\begin{align}
	r(m-k) = E\left[u(n-k)u^{*}(n-m)\right]
	\label{eq:WienerAutocorrelation}
\end{align}
The expectation on the left side of equation %
\ref{eq:ExpandedOptimalityCondition} is the %
\emph{cross correlation} between the filter %
input and the desired response for a lag %
of $-k$ and can be expressed as
\begin{align}
	p(-k) = E\left[u(n-k)d^{*}(n)\right]
	\label{eq:WienerCrossCorrelation}
\end{align}
Substituting equations \ref{eq:WienerAutocorrelation} %
and \ref{eq:WienerCrossCorrelation} into %
\ref{eq:ExpandedOptimalityCondition} %
we get
\begin{align}
	\sum_{m=0}^{N-1}w_{o_{m}}r(m-k)=p(-k),\quad
	\text{for }k=0,1,\cdots,N-1
	\label{eq:WienerHopf}
\end{align}
The system of equations in \ref{eq:WienerHopf} %
define the optimal filter coefficients for the %
equaliser and are named the \emph{Wiener-%
Hopf equations}.
\subsubsection{Solution to the Wiener-Hopf Equations}
Matrix representations of the system of equations in %
\ref{eq:WienerHopf} will let us easily solve for %
the optimum filter coefficients $w_{o_{m}}$. %
Let $\bold{R}$ denote the $N$-by-$N$ correlation %
matrix of the vector of inputs to the optimal filter %
$\bold{u}(n) = [u(n),u(n-1),\cdots,u(n-(N-1))]^T$.
\begin{align}
	\bold{R} = E\left[\bold{u}(n)\bold{u}^{H}(n)\right]
\end{align}
where $^{T}$ denotes the transpose operation and %
$^{H}$ denotes the Hermitian or the complex conjugate %
tranpose operation. Similarly let $\bold{p}$ be the %
$N$-by-$1$ cross correlation vector between %
the inputs and the desired output. 
\begin{align}
	\bold{p} = E\left[\bold{u}(n)d^{*}(n)\right]
\end{align}
Equation \ref{eq:WienerHopf} may now be rewritten %
as
\begin{align}
	\bold{R}\bold{w_{o}} = \bold{p}
	\label{eq:WienerHopfMatrix}
\end{align}
where $\bold{w_{o}}$ denotes the $N$-by-$1$ optimal %
filter coefficient vector $[w_{o_{0}},w_{o_{1}},%
\cdots,w_{o_{N-1}}]^T$. To solve for $\bold{w_{o}}$ %
we simply premultiply both sides of equation %
\ref{eq:WienerHopfMatrix} by $\bold{R}^{-1}$ giving
\begin{align}
	\bold{w_{o}} = \bold{R}^{-1}\bold{p}
	\label{eq:WienerHopfSolution}
\end{align}
\section{Stochastic Gradient and the Least Mean Square}
\subsection{Normalised-Least Mean Square}

\section{Recursive Least Square}
\subsection{Kalman filters}  % Do I want to cover these?


\chapter{System Design}

Here I will develop my system model. Two separate designs were %
developed as part of my proof of concept. The first was a %
computer simulation designed in MATLAB, the second was a %
proof of concept developed on the National Instruments %
Universal Software Radio Platform (USRP) % TODO:include the model no.


\section{Time Invariant System Model}

%TODO: Carefully develop the MATLAB implementation

\section{Time Varying System Model}

%TODO: Carefully develop the MATLAB implementation

%NOTE: Rigourously justify every use of a MATLAB library function.

\section{USRP Implementation of the Time Varying System Model}

%TODO: Carefully develop the USRP implementation


\chapter{Results}
\label{chap:Results}
\section{Time Invariant Simulation Results}
In this section I will be covering my results for the time-invariant model %
as described in chapter \ref{chap:System}. The results will be broken %
down into two sections, first we will examine the LMS algorithm and %
its convergence, then we will examine the NLMS algorithm.

\subsection{Least Mean Square}
\FloatBarrier
The three key measures of the LMS algorithm are it's mean square error, %
mean square deviation, and excess mean square error. %
The mean square error is a measure of how close the estimate is to the %
desired output and is defined as 
\begin{align}
	J = E\left[ \lvert d(n) - y(n) \rvert^{2} \right]
\end{align}
Mean square deviation is a measure of how much error there is %
between the optimal filter coefficients and the estimated %
channel coefficients and is defined as
\begin{align}
	\mathscr{D} = E\left[ \lvert w_{o}(n) - \hat{w}(n) \rvert^{2} \right]
\end{align}
And the excess mean square error is a measure of how much %
difference there is between the mean square error of the %
optimal solution and the estimated solution and is defined as
\begin{align}
	J_{\text{excess}} = J_{o} - J
\end{align}
Figures \ref{fig:LMS-MSE}, \ref{fig:LMS-MSD}, and \ref{fig:LMS-EMSE} %
show the results for mean square error, mean square deviation and %
excess mean square error for the LMS filter under a variety of %
different step sizes $\mu$.
\begin{figure}[ht]
	\includegraphics[width=\textwidth]{./Figures/Results/%
	TimeInvariantLMS/NoDivergence/MeanSquareError.png}
	\caption{Mean Square Error of LMS}
	\label{fig:LMS-MSE}
\end{figure}
\begin{figure}[ht]
	\includegraphics[width=\textwidth]{./Figures/Results/%
	TimeInvariantLMS/NoDivergence/MeanSquareDeviation.png}
	\caption{Mean Square Deviation of LMS}
	\label{fig:LMS-MSD}
\end{figure}
\begin{figure}[ht]
	\includegraphics[width=\textwidth]{./Figures/Results/%
	TimeInvariantLMS/NoDivergence/ExcessMeanSquareError.png}
	\caption{Excess Mean Square Error of LMS}
	\label{fig:LMS-EMSE}
\end{figure}
It's apparent for the figures that as the step size increases the %
rate of convergence increases as well. Recalling the discussion in %
chapter \ref{chap:AdaptiveFiltering} on adaptive filtering, this agrees %
with the intuition of larger step sizes moving more quickly down the error %
surface towards the optimum solution. Drawing our attention to figure %
\ref{fig:LMS-EMSE} it's clear that for larger step sizes the excess %
mean square error is higher in steady state. This also agrees with the %
stochastic gradient theory. When the step size is large, noise will cause %
the LMS filter to overshoot the optimum solution and oscillate around it, %
leading to larger excess mean square error or misadjustment. This confirms %
that given the time invariant system model the LMS filter operating in the %
frequency domain does in fact converge to the optimal solution and that %
the expected tradeoff between convergence time and misadjustment is %
present.
\begin{figure}[ht]
	\includegraphics[width=\textwidth]{./Figures/Results/%
	TimeInvariantLMS/Divergence/MeanSquareError.png}
	\caption{Mean Square Error diverging of LMS}
	\label{fig:LMS-MSE-Diverge}
\end{figure}
\begin{figure}[ht]
	\includegraphics[width=\textwidth]{./Figures/Results/%
	TimeInvariantLMS/Divergence/MeanSquareDeviation.png}
	\caption{Mean Square Deviation diverging of LMS}
	\label{fig:LMS-MSD-Diverge}
\end{figure}
Table \ref{tab:LMS} lists that valid step sizes for $\mu$ range between %
$0$ and $2/(M S_{\text{max}})$, however this is only valid for when %
$M$ is large. Instead we will examine the divergence of the LMS filter %
under the condition that $\mu$ must be between:
\begin{align}
	0 < \mu < \frac{2}{\sigma^{2}}
	\label{eq:LMS-Stability}
\end{align}
Where $\sigma^{2}$ is the variance in the input signal, and can also %
be thought of as the average power of the input signal. This follows %
directly from the conditions imposed upon step size by the gradient %
descent optimisation method. Figures \ref{fig:LMS-MSE-Diverge}, %
\ref{fig:LMS-MSD-Diverge}, \ref{fig:LMS-EMSE-Diverge} are %
show systems that diverge at a $\mu = 0.35$. This occurs %
when the randomly generated fading is very large and reduces %
the upper bound in equation \ref{eq:LMS-Stability} to beneath %
$\mu$.
\begin{figure}[ht]
	\includegraphics[width=\textwidth]{./Figures/Results/%
	TimeInvariantLMS/Divergence/ExcessMeanSquareError.png}
	\caption{Excess Mean Square Error diverging of LMS}
	\label{fig:LMS-EMSE-Diverge}
\end{figure}
\FloatBarrier
\subsection{Normalized Least Mean Square}
\FloatBarrier
Similarly to the LMS filter the NLMS filter also demonstrates convergence %
to the optimal solution as can be seen in figures \ref{fig:NLMS-MSE}, %
\ref{fig:NLMS-MSD}, and \ref{fig:NLMS-EMSE}. It's very apparent when %
comparing the $\mu = 0.05$ lines of the LMS and NLMS filters that the %
NLMS filter has faster convergence. The NLMS filter also has a greater range of %
convergent choices for $\mu$ as can be seen from the legend. This is an agreement %
with equation \ref{eq:SayedNLMSBound}. At higher choices of $\mu$ the NLMS also has %
a much higher misadjustment as can be seen in figure \ref{fig:NLMS-EMSE} which is %
a result of the faster rate of convergence. The regularisation parameter chosen %
for this NLMS experiment is chosen to be $0.05$.
\begin{figure}[ht]
	\includegraphics[width=\textwidth]{./Figures/Results/%
	TimeInvariantNLMS/NoDivergence/MeanSquareError.png}
	\caption{Mean Square Error of NLMS}
	\label{fig:NLMS-MSE}
\end{figure}
\begin{figure}[ht]
	\includegraphics[width=\textwidth]{./Figures/Results/%
	TimeInvariantNLMS/NoDivergence/MeanSquareDeviation.png}
	\caption{Mean Square Deviation of NLMS}
	\label{fig:NLMS-MSD}
\end{figure}
\begin{figure}[ht]
	\includegraphics[width=\textwidth]{./Figures/Results/%
	TimeInvariantNLMS/NoDivergence/ExcessMeanSquareError.png}
	\caption{Excess Mean Square Deviation of NLMS}
	\label{fig:NLMS-EMSE}
\end{figure}
In figures \ref{fig:NLMS-MSE-Diverge}, \ref{fig:NLMS-MSD-Diverge}, and %
\ref{fig:NLMS-EMSE-Diverge} it's apparent that for choices of $\mu$ greater %
than $2$ the NLMS algorithm diverges.
\begin{figure}[ht]
	\includegraphics[width=\textwidth]{./Figures/Results/%
	TimeInvariantNLMS/Divergence/MeanSquareError.png}
	\caption{Mean Square Error Divergence of NLMS}
	\label{fig:NLMS-MSE-Diverge}
\end{figure}
\begin{figure}[ht]
	\includegraphics[width=\textwidth]{./Figures/Results/%
	TimeInvariantNLMS/Divergence/MeanSquareDeviation.png}
	\caption{Mean Square Deviation Divergence of NLMS}
	\label{fig:NLMS-MSD-Diverge}
\end{figure}
\begin{figure}[ht]
	\includegraphics[width=\textwidth]{./Figures/Results/%
	TimeInvariantNLMS/Divergence/ExcessMeanSquareError.png}
	\caption{Excess Mean Square Error Divergence of NLMS}
	\label{fig:NLMS-EMSE-Diverge}
\end{figure}
%TODO: Develop the time invariant results. Compare 
% With existing literature

\section{Time Varying Simulation Results}

%TODO: Develop the time varying results. Compare with 
% existing literature

\section{USRP Implementation Results}


% TODO: Develop the USRP results and compare with the
% simulation results

\chapter{Conclusions and Future Directions}
\label{chap:Conclusion}

A variery of adaptive filtering algorithms have been tested %
and their performance in the learning and decision directed %
schemes measured. Several things can be concluded from %
the results provided in chapter \ref{chap:Results}. Firstly, %
the frequency domain adaptive filtering model described in %
section \ref{sec:TIModel} performs as expected. The single %
variant of the LMS and NLMS algorithms both converge to %
the Wiener solution and exhibit the expected excess %
mean square error increase with step size increase.

Section \ref{sec:TVResults} looked at the behaviour %
of these algorithms under a time varying scenario in %
both training and decision directed operation. The %
time varying results were broken down into %
differeing, SNR's, coherence times, cyclic prefix lengths, %
and decision direction or no decision direction. SNR and %
coherence time were found to have a significant impact %
on the performance of the adaptive filters.

In high SNR conditions, the LMS algorithm was unable %
to converge to the optimal solution even after $1000$ %
symbols, as can be seen in figures \ref{fig:LMS-Long-High-None-Long}, %
\ref{fig:LMS-Medium-High-None-Long}, and %
\ref{fig:LMS-Short-High-None}. The convergence to the %
Wiener solution in this case worsened as coherence %
time was shortened as well. On the other hand the NLMS algorithm %
performed consistently much better in training at high SNR's %
than the LMS algorithm, with both faster rates of convergence and %
very little effect on performance from coherence time. In low %
SNR conditions the LMS algorithm is capable of converging %
to the optimal solution. The NLMS algorithm, even though it %
still has a faster rate of convergence now over fits the solution, %
as can also clearly be seen in the BER results of figures %
\ref{fig:NLMS-BER-Long-None}, \ref{fig:NLMS-BER-Medium-None}, 
and \ref{fig:NLMS-BER-Short-None}. This result %
indicates that the LMS algorithm is only suitable for use in %
lower SNR environments, where the mean square error %
of the Wiener solution is high enough, whereas at high %
SNR's the NLMS algorithm is the clear choice for training and %
intiialisation. In addition at low SNR's the NLMS algorithm %
has a tendency to overfit which may reduce the tracking %
performance of the filters in decision directed operation.

At high SNR's the decision directed performance of the LMS and %
NLMS filters are comparable for long and medium length coherence %
times. For short coherence times it's very clear that at high %
SNR's the NLMS algorithm outperforms the LMS algorithm. It's visible from %
figures \ref{fig:NLMS-Short-High-Directed-Long}, and
\ref{fig:NLMS-Short-High-Directed-Short} that the NLMS algorithm %
in decision directed operation is highly sensitive to sudden disturbances, %
where the peaks in Wiener filter mean square error exist the %
NLMS filter exhibits this staircase behaviour as it steps away %
from the optimal solution, with no indication of recovery. This %
stepping behaviour is also somewhat visible although much less %
obvious in figure \ref{fig:NLMS-Medium-High-Directed-Long}. %
In contrast to this at high SNR's and medium coherence times, %
the LMS algorithm demonstrates some capacity to recover %
back towards the optimal solution as is seen in figure %
\ref{fig:Medium-High-Directed-Long}. At low SNR's the %
LMS algorithm very clearly out performs the NLMS algorithm %
as in all cases of coherence time, the NLMS algorithm almost %
instantly diverges away from the optimal solution, towards %
a mean square error of 1, while the LMS algorithm, even in the %
worst case of short coherence time is able to track the %
Wiener solution for at least $100$ symbols. It's clear then, that %
the LMS algorithm is the clear choice for decision directed tracking %
of the Wiener solution at lower SNR's, the NLMS algorithm probably %
diverges away from the optimal solution so quickly for the same %
reason as it exhibited the staircase behaviour in high SNR operation, in %
the sense that it is very sensitive to incorrect decisions made by the %
decision slicer. 

Since the cyclic prefix samples of every OFDM symbol are dropped. %
Increasing the cyclic prefix of the OFDM symbol may have the %
effect of forcing the adaptive filters to adapt to more change %
between symbols than they otherwise would. Overall the length %
of the cylic prefix seems to have a much more %
muted effect on the system performance than the effect of %
coherence time and SNR (assuming of course that the cyclic %
prefix is longer than the wireless filter length). It's clear however %
from comparing the decision directed NLMS figures with a short %
cyclic prefix that their performance, particularly at low SNR's is %
significantly improved.

The results from section \ref{sec:USRPResults} demonstrate that %
the mean square error convergence on the software defined %
radio does converge in a way that is consistent with the MATLAB %
simulations. It was not possible to examine the decision directed %
performance of the radio implementation with any kind of rigour %
as the unknown channel between the antennae meant that %
a wiener solution cannot be found and as was made clear in section %
\ref{sec:TVResults} the LMS algorithm does not guarantee %
convergence to the optimal solution under a time varying scenario, and %
the NLMS algorithm has a tendency to overfit. Comparing figures %
\ref{fig:NLMS-BER-USRP} with \ref{fig:NLMS-BER-Long-DD-TV}, it is %
clear that this overfitting behaviour is present in the radio implementation %
as well.

A good frequency domain adaptive receiver should require a short %
number of training symbols to initialise itself to the Wiener solution so that %
most of the symbols transmitted can carry data. These results indicate %
that the NLMS algorithm performs significantly better in training convergence %
in all cases with the caveat that the over fitting behaviour seen %
at low SNR needs to be accounted for in some way. The NLMS filter %
also performed better at tracking the time variation in decision directed %
operation at high SNR's for medium to short time coherence, at long %
time coherence the LMS and NLMS algorithms had very similar performance %
under simulation. At low SNR the LMS algorithm is the clear choice %
for decision directed tracking, since it is less influenced by incorrect %
decisions from the decision slicer. It is clear, that even once OFDM %
modulation parameters have been chosen to overcome the %
frequency selectivity of the channel, the choice of adaptive %
filter for good performance is not as simple as choosing the %
NLMS filter, even though it has better performance in a %
time-invariant environment for training. SNR, coherence time, %
and even to some extent the cyclic prefix length must be %
carefully considered and a combination of NLMS and LMS %
filters may be needed depending on how long the continuous %
signal stream will be and how much training data can %
be afforded.

\section{Future Directions}

There are many improvements and areas of study left to %
investigate in this system. Firstly, it is clear that faster %
converging algorithms such as the NLMS are significantly %
better for initialisation to the Wiener solution than slower %
converging algorithms. References \cite{Qureshi77} and %
\cite{Crozier91} suggest that some training sequences %
may have properties that allow for fast convergence %
to the Wiener solution, and so more intelligent choices of %
training sequence is worth investigating. Other types of %
adaptive filters may also be worth looking at such %
as the recursive least square (RLS), given that %
OFDM reduces the channel estimation problem to that %
of a single filter coefficient, it may be computationally %
tractable to look at more computationally complex filters %
such as an affine projection filter or variants on the Kalman %
filter. 

Reference \cite{Wei17} suggests that both time and frequency %
coherence can be exploited for equalisation. Investigating %
the performance of introducing weighted filter coefficient %
interpolation in the frequency domain may significantly %
improve the performance of the adaptive receiver in decision %
directed operation. In addition if the continuous stream of data %
is long, periodic retraining symbols can be transmitted to %
provide corrections to accumulated errors under decision %
directed operation, in combination with frequency domain %
coefficient interpolation, it may not be necessary to %
retransmit these retraining symbols on all the subcarriers. If %
quality of service information can be transmitted upstream %
to the base station and can be responded to in real time, it %
may also be feasible to only selectively resend training %
symbols on subcarriers that are experiencing significant %
departure from the optimal solution. 

In section \ref{sec:TVResults} it was made apparent that the %
NLMS algorithm performed poorly in decision directed operation %
due to the fact that it steps further away from the correct %
decision when the decision slicer makes an incorrect decision. %
The number of incorrect decisions made due to noise, the density %
of the symbol constellation (i.e., 16-QAM, 64-QAM etc.) will both %
have a significant impact on how well these adaptive filters will %
perform while decision directed. This study does not spend any %
time evaluating the relationship between constellation density, %
noise and decision directed performance, and it may be %
important to characterise this as higher constellation density %
may reduce decision directed performance to an impractical %
level at similar $E_b/N_0$, so additional $E_b/N_0$ may be %
required.

% TODO: Develop future study directions for this.

\newpage
\bibliography{./References/references}
\bibliographystyle{IEEETrans}
\begin{appendices}
\chapter{Tx System Model}
\label{app:TxSysModel}
\begin{sidewaysfigure}
	\centering
	\begin{subfigure}{\paperwidth}
		\includegraphics[width=\linewidth]{./Figures/System/%
		TxPart1.png}
		\caption{Tx System Model}
		\label{fig:USRPTx1}
	\end{subfigure}
	\begin{subfigure}{\linewidth}
		\centering
		\includegraphics[width=0.7\linewidth]%
		{./Figures/System/%
		TxPart2.png}
		\caption{Tx System Model}
		\label{fig:USRPTx2}
	\end{subfigure}
\end{sidewaysfigure}
\chapter{Rx System Model}
\label{app:RxSysModel}
\begin{sidewaysfigure}
	\centering
	\begin{subfigure}{\paperwidth}
		\centering
		\includegraphics[width=\linewidth]{./Figures/System/%
		RxPart1.png}
		\caption{Rx System Model}
		\label{fig:USRPRx1}
	\end{subfigure}
	\begin{subfigure}{\linewidth}
		\centering
		\includegraphics[width=0.7\linewidth]%
		{./Figures/System/%
		RxPart2.png}
		\caption{Rx System Model}
		\label{fig:USRPRx2}
	\end{subfigure}
\end{sidewaysfigure}
\end{appendices}
\end{document}
