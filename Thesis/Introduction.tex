\chapter{Introduction}

\section{Motivating Scenario}
The next generation of wireless communication %
dubbed 5G makes promises of a hyper connected %
world with billions of devices coming online % TODO: cite this
connected by the radio network. With this kind %
of communication network in mind, it is easy %
to imagine a scenario in which a cell needs %
to serve many almost static users a steady stream %
of data over a long period of time.% 

\begin{figure}[h!]
	\includegraphics[width=\linewidth]{./Figures/%
	Application_Scenario.png}
	\caption{Application Scenario}
	\label{fig:AppScene}
\end{figure}

Figure \ref{fig:AppScene}. visualises the motivating %
scenario in mind of one base station serving many %
terminals over a bidirectional link. I'll be constraining %
% TODO: Add up-link study to the conclusion about future study
the study done within this report to simply the down-link % 
between the base station and a terminal. This link will %
have a few properties of interest to us. The first is that %
the terminal will be fairly static with very little movement, %
the second, is that the environment is prone to change. We can %
imagine this scenario as being a user, sitting down in a busy %
courtyard watching streaming video on their mobile phone. The %
user equipment is unlikely to move much however the environment %
connecting the user to the base station is prone to change as %
people and vehicles move about. I will model this environment %
as one that will introduce slow time-variations to the wireless %
% Find some examples or citations of this?
channel connecting the station to the user. No more time will be %
spent analysing the validity of this channel model as %
the key focus of this report will be to analyse the %
effectiveness of adaptive filters under these channel %
conditions.

With this scenario in mind this report will be broken %
into the following sections. The rest of this chapter %
will aim to introduce the topics of the wireless %
communication principles used throughout the rest %
of the report, adaptive filtering, orthogonal %
frequency division multiplexing, and wireless %
channel modeling will receive a high level treatment %
here. Chapter 2 will give OFDM a more in-depth %
mathematical discussion. % This will probably be very short
% but I can probably flesh it out if I give it some 
% comparison and look at OOB emissions and the such.
Chapter 3 will give adaptive filtering a similar %
in-depth mathematical treatment. Chapter 4 will study %
channel models and channel modeling techniques. %
Chapter 5 will develop my system model both under %
under simulation and on the software defined radio. %
Chapter 6 will look at the results under simulation %
both on the radio and in MATLAB. Chapter 7 will draw %
some conclusions and make some remarks about future %
directions that can be explored.

\section{Orthogonal Frequency Division Multiplexing %
(OFDM)}

\section{Adaptive Filtering}

\section{Channel Modeling}

