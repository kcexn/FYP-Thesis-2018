\chapter{Orthogonal Frequency Division Multiplexing (OFDM)}
\label{chap:OFDM}
Chapter 1 introduced the idea of OFDM and why it's %
advantageous to use. This chapter will aim to develop %
OFDM mathematically and provide some theoretical insight %
into how and why it works.
\begin{figure}[ht]
	\centering
	\includegraphics[width=\linewidth]{./Figures/Chapter2/%
	OFDM_Transmit.png}
	\caption{OFDM Transmitter block diagram}
	\label{fig:OFDMTransmit}
\end{figure}
\begin{figure}[ht]
	\centering
	\includegraphics[width=\linewidth]{./Figures/Chapter2/%
	OFDM_Receive.png}
	\caption{OFDM Receiver block diagram}
	\label{fig:OFDMReceiver}
\end{figure}
Figure \ref{fig:OFDMTransmit} and \ref{fig:OFDMReceiver} %
depict the block diagram of the OFDM transmitter and receiver.

\section{Discrete Fourier Transform and Fast Fourier Transform}

It's long been known that signals can be overlapped in the %
frequency domain with orthogonal frequency spacing %
allowing for efficient utilisation of the %
frequency spectrum\cite{Chang66}.

A key driver to the efficiency and widespread adoption %
of OFDM is the discrete Fourier transform (DFT) defined %
as\cite{Rao2010}:
\begin{align}
	X^F(k) = \sum_{n=0}^{N-1}x(n)e^{(\frac{-i 2\pi}{N})kn}, %
	k=0,1,\cdots,N-1,\quad N \text{ DFT coefficients}
\end{align}
And the inverse transform is defined as:
\begin{align}
	x(n) = \frac{1}{N}\sum_{k=0}^{N-1}X^{F}(k)%
	e^{(\frac{i 2\pi}{N})kn}, n = 0,1,\cdots,N-1, \quad%
	N \text{ data samples}
	\label{eq:IDFT}
\end{align}
The discrete fourier transform has the orthogonality %
properties necessary for orthogonal spectral allocation. %
To show this I'll follow the development in \cite{Opp99}, %
multiply both sides of eq \ref{eq:IDFT} by %
$e^{-j(\frac{2\pi}{N})rn}$ and summing from $n=0$ %
to $n=N-1$ gives:
\begin{align}
	\sum_{n=0}^{N-1}x(n)e^{(\frac{-i 2\pi}{N})rn} = %
	\sum_{k=0}^{N-1}X^{F}(k)\left[\frac{1}{N} \sum_{n=0}^{N-1} %
	e^{(\frac{i 2\pi}{N})(k-r)n}\right]
\end{align}
evaluating the exponential:
\begin{align}
	\frac{1}{N}\sum_{n=0}^{N-1}e^{(\frac{i 2\pi}{N})(k-r)n}%
	= \begin{cases}
		1,\quad k-r = mN, \quad m\text{ an integer},\\
		0,\quad \text{otherwise.}
	\end{cases}
\end{align}
which expresses the orthogonality of the complex exponentials %
$e^{(\frac{i 2\pi}{N})kn}$ and $e^{(\frac{-i 2\pi}{N})rn}$ for %
integer values of $k$ and $r$. This can be interpreted as the %
complex exponentials forming an orthonormal basis demonstrating %
that %
the DFT and IDFT have the orthogonality properties needed for %
OFDM.

By itself the DFT has computational complexity of $\mathcal{O}(%
N^2)$ as there are $N$ multiplication for each coefficient and %
$N$ coefficients in the DFT. Fast versions of the DFT which %
reduce the computational complexity of the DFT to $\mathcal{O}(%
Nlog(N))$ have been fundamental to the adoption of OFDM.%

The fast Fourier transform (FFT) used in the remainder of %
this report % TODO: thesis? 
is the default FFT provided by MATLAB built on FFTW\cite{FFTW}. %
\section{Cyclic Prefix}

In addition to the FFT a second principal component to the %
operation of OFDM is the cyclic prefix. The development of %
the cyclic prefix here follows that of \cite{Goldsmith05}.

Consider an input to a wireless channel $x[n] = x[0],x[1],\cdots,%
x[N-1]$ of length $N$ and a discrete-time channel with finite %
impulse response (FIR) $h[n] = h[0],h[1],\cdots,h[\mu-1]$ of length %
$\mu$. A cyclic prefix of $x[n]$ is defined as $\{x[N-(\mu-1)], %
x[N-(\mu-2)], \cdots, x[N-1]\}$ which is appended to the beginning %
of the input, the resulting series of samples %
$\hat{x} = x[N-(\mu-1)], %
x[N-(\mu-2)], \cdots, x[N-1], x[0], x[1], \cdots, x[N-1]$, figure %
\ref{fig:CyclicPrefix}. visualises $\hat{x}[n]$.
\begin{figure}[h!]
	\includegraphics[width=\linewidth]{./Figures/Chapter2/%
	CyclicPrefix.png}
	\caption{Appending the Cyclic Prefix}
	\label{fig:CyclicPrefix}
\end{figure}
It's clear from the definition of the new sequence $\hat{x}[n]$ that %
$\hat{x}[n]$ is equivalent to $x[n]_{N}$ from $-(\mu-1) \leq n \leq %
N-1$. Where $x[n]_{N}$ denotes the sequence $x[n]$ modulo $N$, that %
is to say the sequence $x[n]$ is periodic in $N$. Given $\hat{x}[n]$ %
as the input to the channel and the channel response begin $h[n]$. %
The channel output is:
\begin{align}
	y[n] =& \hat{x}[n]*h[n]\\
	     =& \sum_{k=0}^{\mu-1}h[k]\hat{x}[n-k]\\
	     =& \sum_{k=0}^{\mu-1}h[k]x[n-k]_{N}\\
	     =& x[n]_{N}\circledast h[n]
\end{align}
where $\circledast$ represents a circular convolution. %

The convolution theorem states that the convolution %
of two  signals is equivalent to point wise %
multiplication in the frequency domain\cite{MathWorldConvolution%
Theorem}. In the special case of the DFT, the convolution theorem %
applies under circular convolution i.e.
\begin{align}
	Y[i] = \text{DFT}\{y[n] = x[n]\circledast h[n]\} = X[i]H[i]	
	\label{eq:DFTConvolutionTheorem}
\end{align}
This powerful result allows for the easy equalisation of the %
received %
signal $y[n]$ in the frequency domain, as the sent signal $X[i]$ can %
retrieved with a simple division.
\begin{align}
	X[i] = \frac{Y[i]}{H[i]}
	\label{eq:ZeroForcing}
\end{align}
