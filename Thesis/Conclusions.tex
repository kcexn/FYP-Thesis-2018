\chapter{Conclusions and Future Directions}
\label{chap:Conclusion}

A variery of adaptive filtering algorithms have been tested %
and their performance in the learning and decision directed %
schemes measured. Several things can be concluded from %
the results provided in chapter \ref{chap:Results}. Firstly, %
the frequency domain adaptive filtering model described in %
section \ref{sec:TIModel} performs as expected. The single coefficient %
variant of the LMS and NLMS algorithms both converge to %
the Wiener solution and exhibit the expected excess %
mean square error increase with step size increase.

Section \ref{sec:TVResults} looked at the behaviour %
of these algorithms under a time varying scenario in %
both training and decision directed operation. The %
time varying results were broken down into %
differing SNR's, coherence times, cyclic prefix lengths, %
and decision direction or no decision direction. SNR and %
coherence time were found to have a significant impact %
on the performance of the adaptive filters.

In high SNR conditions, the LMS algorithm was unable %
to converge to the optimal solution even after $1000$ %
symbols, as can be seen in figures \ref{fig:LMS-Long-High-None-Long}, %
\ref{fig:LMS-Medium-High-None-Long}, and %
\ref{fig:LMS-Short-High-None}. The convergence to the %
Wiener solution in this case worsened as coherence %
time was shortened as well. On the other hand the NLMS algorithm %
performed consistently much better in training at high SNR's %
than the LMS algorithm, with both faster rates of convergence and %
very little effect on performance from coherence time. In low %
SNR conditions the LMS algorithm is capable of converging %
to the optimal solution. The NLMS algorithm, even though it %
still has a faster rate of convergence now over fits the solution, %
as can also clearly be seen in the BER results of figures %
\ref{fig:NLMS-BER-Long-None}, \ref{fig:NLMS-BER-Medium-None}, 
and \ref{fig:NLMS-BER-Short-None}. This overfitting of the NLMS %
algorithm may induce a cost to the mean square error in the decision %
directed region as illustrated in figures \ref{fig:Combination-Long-Long} %
to \ref{fig:Combination-Short-Short}. These results %
indicate that the LMS algorithm is only suitable for use in %
lower SNR environments, where the mean square error %
of the Wiener solution is high enough, whereas at high %
SNR's the NLMS algorithm is the clear choice for training and %
intialisation. In addition at low SNR's the NLMS algorithm %
has a tendency to overfit which may reduce the tracking %
performance of the filters in decision directed operation.

At high SNR's the decision directed performance of the LMS and %
NLMS filters are comparable for long and medium length coherence %
times. For short coherence times it is very clear that at high %
SNR's the NLMS algorithm outperforms the LMS algorithm. It is visible from %
figures \ref{fig:NLMS-Short-High-Directed-Long}, and
\ref{fig:NLMS-Short-High-Directed-Short} that the NLMS algorithm %
in decision directed operation is highly sensitive to sudden disturbances, %
where the peaks in Wiener filter mean square error exist the %
NLMS filter exhibits this staircase behaviour as it steps away %
from the optimal solution, with no indication of recovery. This %
stepping behaviour is also somewhat visible although much less %
obvious in figure \ref{fig:NLMS-Medium-High-Directed-Long}. %
In contrast to this at high SNR's and medium coherence times, %
the LMS algorithm demonstrates some capacity to recover %
back towards the optimal solution as is seen in figure %
\ref{fig:Medium-High-Directed-Long}. At low SNR's the %
LMS algorithm very clearly out performs the NLMS algorithm %
as in all cases of coherence time, the NLMS algorithm almost %
instantly diverges away from the optimal solution, towards %
a mean square error of 1, while the LMS algorithm, even in the %
worst case of short coherence time is able to track the %
Wiener solution for at least $100$ symbols. It's clear then, that %
the LMS algorithm is the clear choice for decision directed tracking %
of the Wiener solution at lower SNR's, the NLMS algorithm probably %
diverges away from the optimal solution so quickly for the same %
reason as it exhibited the staircase behaviour in high SNR operation, in %
the sense that it is very sensitive to incorrect decisions made by the %
decision slicer. 

Since the cyclic prefix samples of every OFDM symbol are dropped. %
Increasing the cyclic prefix of the OFDM symbol may have the %
effect of forcing the adaptive filters to adapt to more change %
between symbols than they otherwise would. Overall the length %
of the cylic prefix seems to have a much more %
muted effect on the system performance than the effect of %
coherence time and SNR (assuming of course that the cyclic %
prefix is longer than the wireless filter length). It's clear however %
from comparing the decision directed NLMS figures with a short %
cyclic prefix that their performance, particularly at low SNR's is %
significantly improved.

The results from section \ref{sec:USRPResults} demonstrate that %
the mean square error convergence on the software defined %
radio does converge in a way that is consistent with the MATLAB %
simulations. It was not possible to examine the decision directed %
performance of the radio implementation with any kind of rigour %
as the unknown channel between the antennae meant that %
a Wiener solution was not evaluated due to the reasons explained in %
section \ref{sec:USRPMultipathChannelSimulation}. This problem was %
exacerbated by the fact that the LMS algorithm does not guarantee %
convergence to the optimal solution under a time varying scenario, and %
the NLMS algorithm has a tendency to overfit at low SNR's. Comparing figures %
\ref{fig:NLMS-BER-USRP} with \ref{fig:NLMS-BER-Long-DD-TV}, it is %
clear that this overfitting behaviour is present in the radio implementation %
as well.

A good frequency domain adaptive receiver should require a short %
number of training symbols to initialise itself to the Wiener solution so that %
most of the symbols transmitted can carry data. These results indicate %
that the NLMS algorithm performs significantly better in training convergence %
in all cases with the caveat that the over fitting behaviour seen %
at low SNR needs to be accounted for in some way. The NLMS filter %
also performed better at tracking the time variation in decision directed %
operation at high SNR's for medium to short time coherence, at long %
time coherence the LMS and NLMS algorithms had very similar performance %
under simulation. At low SNR the LMS algorithm is the clear choice %
for decision directed tracking, since it is less influenced by incorrect %
decisions from the decision slicer. It is clear, that even once OFDM %
modulation parameters have been chosen to overcome the %
frequency selectivity of the channel, the choice of adaptive %
filter for good performance is not as simple as choosing the %
NLMS filter, even though it has better performance in a %
time-invariant environment for training. SNR, coherence time, %
and even to some extent the cyclic prefix length must be %
carefully considered and a combination of NLMS and LMS %
filters may be needed depending on how long the continuous %
signal stream will be and how much training data can %
be afforded.

\section{Future Directions}

There are many improvements and areas of study left to %
investgate in this system. Firstly, a severe disadvantage %
to this kind of filtering is that each time step during training %
uses $N*\text{log}\left( M \right)$ bits of potential data, where %
$N$ is the FFT length and M is the size of the QAM constellation, i.e. %
if no training was required an additional $N*\text{log}\left( M \right)$ %
bits of data would be sent. It is clear then, that %
it is beneficial for the training sequence to be as short %
as possible to maximise the amount of data throughput. %
References \cite{Qureshi77} and \cite{Crozier91} suggest %
that some training sequences may have properties that %
allow for fast convergence to the Wiener solution. %
An investigation can be conducted into intelligent %
choices of training sequences that may offer %
fast convergence and significantly improve the %
data throughput of the system.

Other types of filters exist, in particular the class %
of recursive least square (RLS) filters have been %
left unexplored in this study so far. Given that %
OFDM reduces the channel estimation problem %
to that of a single filter coefficient in the frequency %
domain for each subcarrier, it may be computationally %
tractable to look at more computationally complex filters %
such as affine projection filters or variants on the Kalman %
filter.

Reference \cite{Wei17} suggests that both time and frequency %
coherence can be exploited for equalisation. Investigating %
the performance of introducing weighted filter coefficient %
interpolation in the frequency domain may significantly %
improve the performance of the adaptive receiver in decision %
directed operation. In addition if the continuous stream of data %
is long, periodic retraining symbols can be transmitted to %
provide corrections to accumulated errors under decision %
directed operation, in combination with frequency domain %
coefficient interpolation, it may not be necessary to %
retransmit these retraining symbols on all the subcarriers. If %
quality of service information can be transmitted upstream %
to the base station and can be responded to in real time, it %
may also be feasible to only selectively resend training %
symbols on subcarriers that are experiencing significant %
departure from the optimal solution. 

In section \ref{sec:TVResults} it was made apparent that the %
NLMS algorithm performed poorly in decision directed operation %
due to the fact that it steps further away from the correct %
decision when the decision slicer makes an incorrect decision. %
The number of incorrect decisions made due to noise, the density %
of the symbol constellation (i.e., 16-QAM, 64-QAM etc.) will both %
have a significant impact on how well these adaptive filters will %
perform while decision directed. This study does not spend any %
time evaluating the relationship between constellation density, %
noise and decision directed performance, and it may be %
important to characterise this as higher constellation density %
may reduce decision directed performance to an impractical %
level at similar $E_b/N_0$, so additional $E_b/N_0$ may be %
required.

My experiments with the USRP never got around to approximating %
the Wiener solution. Partially this was due to the learning curve of %
working with the NI LabView environment, and partially this was %
due to the fact that the unknowns introduced by the %
radio wireless channel meant that not only would an approximation %
of the expectation in equation \ref{eq:WienerAutocorrelation} as %
was done in the time-varying MATLAB simulation, but also an approximation to the %
expectation in \ref{eq:WienerCrossCorrelation}, a future improvement %
to the radio simulation model would be to make this approximation and %
more rigourously compare the radio model to the MATLAB simulation.
