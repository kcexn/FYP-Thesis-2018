\chapter{Introduction}
\label{ch:Introduction}
\section{Motivating Scenario}
\label{sec:MotivatingScenario}
The next generation of wireless communication %
dubbed 5G makes promises of a hyper connected %
world with billions of devices coming online % TODO: cite this
connected by the radio network. With this kind %
of communication network in mind, it is easy %
to imagine a scenario in which a cell needs %
to serve many almost static users a steady stream %
of data over a long period of time.% 

\begin{figure}[h!]
	\includegraphics[width=\linewidth]{./Figures/%
	Application_Scenario.png}
	\caption{Application Scenario}
	\label{fig:AppScene}
\end{figure}

Figure \ref{fig:AppScene}. visualises the motivating %
scenario in mind of one base station serving many %
terminals over a bidirectional link. I'll be constraining %
% TODO: Add up-link study to the conclusion about future study
the study done within this report to simply the down-link % 
between the base station and a terminal. This link will %
have a few properties of interest to us. The first is that %
the terminal will be fairly static with very little movement, %
the second, is that the environment is prone to change. We can %
imagine this scenario as being a user, sitting down in a busy %
courtyard watching streaming video on their mobile phone. The %
user equipment is unlikely to move much however the environment %
connecting the user to the base station is prone to change as %
people and vehicles move about. I will model this environment %
as one that will introduce slow time-variations to the wireless %
% Find some examples or citations of this?
channel connecting the station to the user. No more time will be %
spent analysing the validity of this channel model as %
the key focus of this report will be to analyse the %
effectiveness of adaptive filters under these channel %
conditions.

With this scenario in mind this report will be broken %
into the following sections. The rest of this chapter %
will aim to introduce the topics of the wireless %
communication principles used throughout the rest %
of the report, adaptive filtering, orthogonal %
frequency division multiplexing, and wireless %
channel modeling will receive a high level treatment %
here. Chapter 2 will give OFDM a more in-depth %
mathematical discussion. % This will probably be very short
% but I can probably flesh it out if I give it some 
% comparison and look at OOB emissions and the such.
Chapter 3 will give adaptive filtering a similar %
in-depth mathematical treatment. Chapter 4 will study %
channel models and channel modeling techniques. %
Chapter 5 will develop my system model both under %
under simulation and on the software defined radio. %
Chapter 6 will look at the results under simulation %
both on the radio and in MATLAB. Chapter 7 will draw %
some conclusions and make some remarks about future %
directions that can be explored.

\section{Orthogonal Frequency Division Multiplexing %
(OFDM)}

The simplest way to modulate signals over the wireless %
channel is to use some form of single carrier modulation. %
Single carrier modulation simply takes input data, separates %
it into in-phase and quadrature components if necessary, pulse %
shapes the signal, up converts to the carrier frequency and %
transmits over the wireless channel.
%TODO: add figure to depict single carrier methodology

This method of modulation has some drawbacks. %TODO: find some 
% references that list some drawbacks of single carrier modulation.
The main drawbacks of interest to us will be the complexity of %
channel equalisation. The wireless propagation medium can be %
modeled as a finite impulse response filter %TODO: cite this
and so as the filter length increases estimating, %
the channel becomes an increasingly more complex task. %
A modulation technique that helps to overcome this challenge %
is orthogonal frequency division multiplexing or OFDM. %
OFDM works differently to traditional single carrier %
modulation in the sense that it separates the modulation %
bandwidth into $N$ subcarriers each with a subband bandwidth of %
$\frac{B_total}{N}$. A key advantage of these subcarriers in %
relation to the scenario in \ref{sec:MotivatingScenario} is that %
the the subbands of the OFDM subcarriers can be designed to %
be narrower than the coherence bandwidth of the time %
varying fading channel. More on coherence bandwidth will %
be covered in chapter \ref{chap:ChannelModeling}

%TODO: add figure depicting the serial to parallel OFDM

A cyclic prefix can be added to the beginning of each OFDM %
symbol which gives it the nice mathematical properties %
which make equalisation simple regardless of the length %
of the channel filter. In addition, the parallel nature %
of the signal means that signal processing can happen %
on each subband independently and can reduce the bandwidth %
requirements for signal processing electronics.

The details of OFDM will be developed in Chapter 2. 
%TODO: fix chapter numbers
Where the mathematical construction of OFDM will be
explored as well as the nature of the cyclic prefix.

\section{Channel Modeling}

The wireless communication channel is a %
hostile environment. There are three %
major types of impairments that affect %
wireless communication

\begin{itemize}
	\item{Path loss}
	\item{Thermal electronic noise}
	\item{Fading}
\end{itemize}

Path loss is an impairment that affects %
all forms of electronic communication and is %
caused by the natural decay in amplitude %
that electromagnetic waves undergo as they %
propagate through free space. Path loss %
is proportional to $\frac{1}{d^2}$ where %
$d$ is the distance between the transmitter %
and receiver.

Thermal electronic noise is the noise %
caused by the electronics at the %
transmitter and receiver. The measure %
of thermal noise that is important %
will be the ratio between average %
signal power and average noise power %
commonly referred to as signal-to-%
noise ratio (SNR). The noise is %
commonly fixed by the temperature %
of operation for the electronics. %
The noise power spectral density %
is defined as %

\begin{align}
	N_{0} = \kappa T
\end{align}

where $\kappa$ is the boltzmann %
constant and $T$ is the temperature %
of operation in Kelvin. Noise %
power is typically relative constant %
in a given system and is defined as

\begin{align}
	\sigma^{2} = BT
\end{align}

where $B$ is the bandwidth of the %
signal of interest.

Fading is a stochastic impairment %
that is unique to wireless %
channels and is caused by obstructions %
in the surrounding environment, %
relative movement between the %
transmitter and receiver %
or movement of objects between %
the transmitter and receiver. The %
combination of movement and %
obstructions cause the wave to %
reflect and scatter throughout %
throughout the environment of %
interest before arriving %
at the receiver. This is a %
severe impairment to the wireless %
channel and is central to my %
analysis of adaptive filtering. So %
Chapter \ref{chap:ChannelModeling} %
will develop this in much more detail.

\section{Adaptive Filtering}

In the time varying scenario described earlier in %
section \ref{sec:MotivatingScenario} an single %
static estimate of the wireless channel impairments %
will not give the best equalisation performance as %
the channel will change. Adaptive filters offer a %
a way to track the changes in the wireless channel %
over time so that the optimal solution can be %
followed providing the best equalisation performance %
and minimising the number of bit errors in the received %
bit stream. The filters studied in this report will be %
the least mean square (LMS) filter and the %
normalized least mean square (NLMS) filter. Both %
are considered a member of the class of filters known %
as stochastic gradient filters which minimise mean square %
error by finding the stationary point of the paraboloid %
that defines the mean square error.
\begin{figure}[ht]
	\centering
	\includegraphics[width=0.4\textwidth]{./Figures/%
Circular_Paraboloid_Quadric.png}
	\caption{Circular parabaloid in three dimensions 
	\cite{Paraboloid12}}
\end{figure}

%TODO: Do some reading to find a good way to motivate Adaptive
% filters. Want to compare it to static filters at the 
% very least. Maybe even want to introduce some filter
% theory full stop.
