\chapter{System Design}
\label{chap:System}
The system model is developed in three stages. %
The first two stages were developed in MATLAB, %
and consist of a time invariant model %
for simulating the convergence behaviour of %
the LMS algorithms, and a time varying %
model for showing the tracking performance %
in the environment proposed in section %
\ref{sec:MotivatingScenario}. The %
final stage was developed in LabView on the %
National Instruments Universial Software Radio %
Platform (USRP) model 2943R and presents %
a hardware proof of concept for adaptive %
filtering in slowly time varying channels. %
This section will be dedicated to developing %
the system models. Chapter \ref{chap:Results} %
will develop the results that I collected %
and compare them. 

\section{Time Invariant System Model}
\label{sec:TIModel}
\FloatBarrier
\begin{figure}[ht]
	\includegraphics[width=\textwidth]{./%
	Figures/System/SystemModel.png}
	\caption{System Model}
	\label{fig:SysModel}
\end{figure}
Figure \ref{fig:SysModel} depicts the general %
outline of the system model that has been %
developed. This section will focus on %
developing all the components that will %
be common to all the later sections as %
well as section specific components.
\FloatBarrier
The primary aim of the time invariant %
model was to establish the convergence %
of the LMS algorithm with an OFDM %
transmission scheme, equalising in the %
\emph{frequency} domain.
\begin{figure}[ht]
	\includegraphics[width=\textwidth]{./%
	Figures/System/RxEqualisation.png}
	\caption{Added equalisation step}
	\label{fig:RxEqualiser}
\end{figure}
Figure \ref{fig:RxEqualiser} illustrates %
the location of the equaliser in this model, %
immediately after the FFT and before the %
parallel to serial step. By designing %
the OFDM subcarrier bandwidth to be %
significantly less than the coherence %
bandwidth of the fading channel, a simple %
version of the LMS filter with only one %
filter coefficient can be assigned to %
each subcarrier for equalisation as %
illustrated in figure \ref{fig:Frequency%
LMS}.
\FloatBarrier
\begin{figure}[ht]
	\centering
	\includegraphics[width=0.5\textwidth]{./%
	Figures/System/FrequencyDomainLMS.png}
	\caption{Single Tap LMS on each Sub%
	carrier}
	\label{fig:FrequencyLMS}
\end{figure}
\FloatBarrier
\subsection{Information Source}
The information source needs to be selected to %
model a believable source of data. Selecting %
values randomly with equal probability from %
an alphabet of possible transmitted symbols %
is a common way to generate this information %
source \cite{Jer00}. For binary data, this %
can be modelled as a Bernoulli distribution. %
The MATLAB model for the computation %
simulations use the in-built \mintinline{MATLAB}{randi} %
\cite{randi} function which generates %
a random vector $\mathbf{X}$ where %
each random variable $X$ can take values %
from a user defined set of integers and is %
uniformly distributed. In the special %
case of the Bernoulli distribution %
the set of integers can be defined as $\{0,1\}$.

\subsection{QPSK Modulation and Demodulation}
\FloatBarrier
QPSK or Quadrature Phase Shift Keying is a popular %
modulation scheme. I've chosen to use it primarily %
for its simplicity but also because it contains both
an in-phase and a quadrature component, which are %
the elements required to construct %
Quadrature Amplitude Modulation (QAM) %
which is the current modulation scheme employed %
in 4G systems \cite{Rumney13}
and are seriously being considered for 5G systems.
\begin{figure}[ht]
	\centering
	\includegraphics[width=0.7\textwidth]{./%
	Figures/System/QPSKScatterplotWithA%
	WGN.png}
	\caption{QPSK Constellation, red stars %
	indicate constellation points, blue %
	points are due to noise}
	\label{fig:QPSKConstellation}
\end{figure}

Figure \ref{fig:QPSKConstellation} shows what %
a QPSK constellation looks like and how a %
received signal might appear when AWGN has %
been added to it. To normalise that average %
symbol energy to 1 the constellation points %
are defined at $\{\frac{\sqrt{2}}{2} + \frac{%
\sqrt{2}}{2}i, \frac{\sqrt{2}}{2} - %
\frac{\sqrt{2}}{2}i, -\frac{\sqrt{2}%
}{2} + \frac{\sqrt{2}}{2}i, %
-\frac{\sqrt{2}}{2} - \frac{\sqrt{2}%
}{2}i\}$ such that when the average %
energy is evaluated
\begin{align}
	E\left[E_s\right] = \frac{1}{4} \sum_{k=1}^{4} %
	\lvert E_{s_{k}} \rvert = 1
\end{align}
where $E_{s_{k}}$ is the symbol energy of the the %
$k\text{th}$ element in the symbol set. A QPSK %
modulation scheme has a symbol set of size $4$. %
So is capable of transmitting $log_{2}(4)$ bits %
per symbol. In general a M-QAM modulation scheme %
is able to transmit $log_{2}(M)$ bits per symbol. %
A gray mapping between pairs of bits to the QPSK %
constellation needs to be chosen so that symbol %
errors that are due to the received symbol mapping %
incorrectly to a neighbouring symbol than what was %
desired will introduce only one bit error instead %
of multiple \cite{Sklar01,Goldsmith05}.

MATLAB's Communications System Toolbox offers %
a system object name \mintinline{MATLAB}{comm.QPSKModulator} %
\cite{QPSKModulator} which %
will perform the gray mapping of pairs of bits in a %
bit stream to QPSK constellation points and perform %
the energy normalisation.

Paired with the \mintinline{MATLAB}{comm.QPSKModulator} MATLAB also %
provides a \mintinline{MATLAB}{comm.QPSKDemodulator} \cite{QPSKDemodulator} %
which performs %
performs maximum likelihood detection and maps %
the received QPSK constellation back to a bit %
stream.

\subsection{OFDM}

OFDM has been covered in some detail in chapter %
\ref{chap:OFDM} so here I'll focus on the simulation %
implementation. MATLAB's communication systems %
toolbox provides an \mintinline{MATLAB}{comm.OFDMModulator} %
\cite{OFDMModulator} which %
takes in an input stream of symbols and performs %
an IFFT followed by an IFFT shift to place the DC %
components in the centre. It also prepends the %
last $\mu$ samples of the output of the IFFT to %
generate a cyclic prefix and allows the use %
of a DC Null, the DC null will be relevant in %
our discussion on the USRP model.

\subsection{AWGN}

The power of additive white gaussian noise $\sigma_n^{2}$ %
was defined relative to the transmit signal power, through %
the following evaluation

\begin{minted}{MATLAB}
powerdB = 10*log10(var(OFDM_Symbol));
snr = EbNo + 10*log10(BitsPerSymbol);
noiseVar = 10\textasciicircum(0.1*(powerdB-snr));
\end{minted}

here \mintinline{MATLAB}{var(OFDM_Symbol)} is being evaluated %
as 
\begin{align}
	\frac{1}{N} %
	\sum_{n=0}^{N-1} \lvert\lvert %
	\text{\mintinline{MATLAB}{OFDM_Symbol}}%
	(n) \rvert\rvert^{2}
\end{align}

Where $N$ is the length of the FFT. \mintinline{MATLAB}{EbNo} is a %
variable which defines the normalised bit energy %
to noise power spectral density and is % TODO: Cite EbNo
expressed as \cite{Sklar01}
\begin{align}
	\frac{E_b}{N_0} = \frac{S}{R_b} \div 
	\frac{\sigma_n^{2}}{B}
\end{align}
where $S$ is the average received signal power, %
$R_b$ is the bit rate, $\sigma_n^{2}$ is the %
total additive white gaussian noise power and %
$B$ is the total signal bandwidth. The relation %
between SNR and $E_b/N_0$ is as follows
\begin{align}
	\frac{S}{\sigma_n^{2}} = \frac{E_b}{N_0}
	\frac{R_b}{B}
\end{align}
A modulation scheme such as QPSK doubles the bit %
rate relative to a binary scheme such as on off %
keying or binary phase shift keying (BPSK), so %
SNR needs to be multipled by $2$ for the different %
schemes to experience the same normalised %
signal to noise ratio, given the same signal bandwidth. %
In log units multiplication %
is evaluated as addition as can be seen above.

MATLAB's Commmunications System Toolbox system object %
\mintinline{MATLAB}{comm.AWGNChannel} \cite{AWGNChannel} takes an %
input stream of %
symbols and generates complex gaussian noise with %
zero mean and defined variance. This noise is then %
summed with the samples in the stream to make a noisy %
received signal. In the case of the system model shown in %
figure \ref{fig:SysModel}, the power of the complex %
additive white gaussian noise calculated based on the transmit %
power before channel impairments. This is because the Rayleigh %
fading channel impairment does not affect all subbands equally as %
it is a frequency selective effect, so setting the noise power based %
on the post channel impairment signal would disproportinately affect %
the subcarriers that are experiencing deep fades.

\subsection{Channel Impairment}
\label{sec:TIChannelImpairment}
The channel impairment designed for the time invariant model %
will be different to the channel impairment designed for the time varying model. %
Here the channel impairment will be chosen as a random complex gaussian that %
defines the fading in the frequency domain. For the purpose of testing the convergence %
of the LMS algorithm, a flat fading scenario is introduced where each subcarrier %
is multiplied by the same coefficient $w$ as illustrated in figure \ref{fig:TIChannel}.
\begin{figure}[ht]
	\includegraphics[width=\textwidth]{./Figures/System/%
	ChannelTimeInvariant.png}
	\caption{Time Invariant Flat Fading Channel Model}
	\label{fig:TIChannel}
\end{figure}
By introducing AWGN to the system after channel impairment and then %
applying the LMS individually to each subcarrier in the frequency domain, %
ensemble averages can be taken for differing channel conditions simply by %
increasing the number of subcarriers. A subcarrier count of $1024$ was %
chosen to generate the time invariant results discussed in the next chapter. %

\subsection{Equalisation}
\label{subsec:Equalisation}
As illustrated in figure \ref{fig:FrequencyLMS}, a frequency domain %
LMS scheme was made with a single tap variant of the LMS algorithm on %
each subcarrier. As discussed in previous chapters, this can be done %
because the subcarrier bandwidths have been designed to be significantly %
narrower than the coherence bandwidth, so can be approximated with a %
single filter coefficient. This approximation has been made explicit in this %
time invariant model as described in section \ref{sec:TIChannelImpairment}. %

The single tap LMS approximation follows closely from the development of the %
LMS filter in section \ref{sec:LMS}, in particular we take equation \ref{eq:SimplifiedLMSUpdate} %
and replace the vector of coefficients $\hat{\mathbf{w}}(n+1)$ and $\hat{\mathbf{w}}(n)$ with scalar %
versions, and the vector of filter inputs $\mathbf{u}(n)$ with a scalar input. Giving the %
new expression
\begin{align}
	\hat{w}(n+1) = \hat{w}(n) + \mu u(n) e^{*}(n)
	\label{eq:ScalarLMS}
\end{align}
This version of the LMS update is the computationally simplest %
the LMS can be with two multiplications and one addition. %
The constant computational complexity of the designed LMS means %
that the complexity of this filtering scheme grows with the %
the size of the FFT needed to ensure that the subcarrier bandwidth %
is narrower than the coherence frequency of the fading channel, and %
has computational complexity $\mathcal{O}(N log(N))$. This scalar version %
of the LMS filter is substituted into the LMS blocks shown in figure %
\ref{fig:FrequencyLMS} at the receiver.

In a similar fashion the NLMS filter can be developed with a scalar equation %
\begin{align}
	\hat{w}(n+1) = \hat{w}(n) + \frac{\mu}{\lvert u(n) \rvert^{2}} %
	u(n) e^{*}(n)
	\label{eq:ScalarNLMS}
\end{align}
and substitued into the LMS blocks in figure \ref{fig:FrequencyLMS} to %
evaluate the performance of the NLMS filter.

\section{Time Varying System Model}

The time varying system model only contains two key differences to the %
time invariant model. The first is the introduction of the time-varying Rayleigh %
and Rician channel models, these channel models have been simulated using %
the filtered white gaussian noise method. The Rayleigh and Rician fading %
channels introduce design constraints to the model not present in the %
simplified time invariant model examined in section %
\ref{sec:TIModel} regarding subcarrier bandwidth and coherence %
frequency as well as the OFDM symbol time and channel %
coherence time. The second key difference is the introduction %
of a decision directed scheme for the LMS %
filter to examine its tracking performance after achieving %
the optimal Wiener solution.

\subsection{Filtered White Gaussian Noise Channel Simulation}

The time varying fading channel can be broken down into three %
steps. The power delay needs to be defined, using the %
power delay profile the average powers of each multipath %
coefficient can be defined. Once this is done the time varying %
component of the fading channel needs to be introduced, %
the method of choice for this simulation will be one of filtered %
white gaussian noise. 
\subsubsection{The Power Delay Profile}

The power delay profile is critical in scaling the average power %
of each multipath coefficient. In figure \ref{fig:ChannelFilterSimulation} %
an illustration of the simulation mechanism by which a single channel %
coefficient process can be generated, $\sigma_{n}$ represents %
the scaling factor to ensure that the $n\text{th}$ coefficient %
of the multipath channel $h(\tau_{n};t)$ has average power %
$\sigma_{n}^{2}$. In the simplest form of an uncorrelated %
tap-gain model \cite{Jer00} 
\begin{align}
	E\left[\lvert h(\tau_{n};t) \rvert^{2} \right] = %
	\sigma_{n}^{2} = T^{2}p(nT)
	\label{eq:TapPower}
\end{align}

where $p(nT)$ is the sampled version of a diffuse power %
delay profile where $T$ is the sampling period. Or of a %
discrete multipath channel where the multipath coefficients %
are symbol spaced. For the purpose of my investigation I %
will not consider the case of a discrete multipath channel %
where the multipath channel coefficients are not symbol %
spaced. In general the latter case introduces the need %
to bandlimit the channel filter and interpolate the coefficients. %
This is usually done with an ideal bandlimiting filter %
\cite{Jer00}, and introduces additional complexity that %
is unnecessary for evaluating the effectiveness of %
adaptive equalisation.

\subsubsection{Filtered White Gaussian Noise}

In chapter \ref{chap:ChannelModeling} the time-varying nature of %
the fading channel was established as being closely related to the %
doppler shift of the signal and the maximum doppler frequency $f_d$. %
The doppler spectrum chosen for simulation is the Jakes spectrum as %
defined in equation \ref{eq:JakesSpectrum} as it is a popular spectrum %
widely used in the literature. %TODO: Cite the literature

The filtered white gaussian noise approach aims to recreate this %
time varying effect by directly filtering the random noise representing %
the scattering with a doppler spectrum in the frequency domain\cite{%
MIMO-OFDM10, Iskander}. 
\begin{figure}[ht]
	\includegraphics[width=\textwidth]{./Figures/System/%
	FilteringProcess.png}
	\caption{Channel Filter Tap Generation}
	\label{fig:ChannelFilterSimulation}
\end{figure}
Figure \ref{fig:ChannelFilterSimulation} illustrates the process %
used to generate each individual coefficient in the discrete %
multipath channel. $z(t)$ is an independent and identically %
distributed (i.i.d) time series where at each timestep $t$, % 
$z(t)\sim\mathcal{N}(0,1)$ (is normally distributed %
with mean $0$ and standard deviation $1$), $S(\nu)$ is the %
Jakes doppler spectrum as described in equation %
\ref{eq:JakesSpectrum} and $\sigma_n$ is %
a scaling factor to ensure the average power of the coefficient %
matches that of the power delay profile.

MATLAB's Communications System Toolbox provides a system object %
\mintinline{MATLAB}{comm.RayleighChannel} \cite{RayleighChannel}, %
which implements the filtered %
white gaussian noise method of simulation as described in 
\cite{Iskander}. Rician distributions are implemented slightly %
differently where the Rician coefficients are further scaled by a %
deterministic component with a specular to %
scattered power ratio $K$ as described in equation %
\ref{eq:KFactor}. The Rician fading component is scaled by %
\begin{align}
	h(\tau;t) = \sigma_n\left[ \frac{z_{k}\left[n\right]}
	{\sqrt{K + 1}} + 
	\sqrt{\frac{K}{K+1}}\right]
\end{align}
Where $z_{k}$ is the filtered noise output of the doppler filter 
\cite{Iskander} immediately after the normalization step in figure %
\ref{fig:ChannelFilterSimulation}. Rician fading channels are %
are implemented in MATLAB following this methodology in %
\mintinline{MATLAB}{comm.RicianChannel} \cite{RicianChannel}.

\subsection{Decision Directed Performance}

It's important to evaluate the performance of the time varying model %
in a decision directed scheme. That is, after a given %
training period the Wiener solution has been found using an adaptive %
filtering approach. In particular it is of interest how long the %
adaptive filtering approach can maintain a solution that is near %
optimal without additional further training %
information. A decision directed model is one where the %
output of the equaliser is fed into the maximum likelihood %
decision slicer, and the output of the decision slicer is fed back %
into the equaliser as the desired %
output in the update equation (\ref{eq:ScalarLMS} and \ref{eq:%
ScalarNLMS}), where $e(n) = d(n) - y(n)$ as illustrated in figure %
\ref{fig:DDLMS}.
\begin{figure}[ht]
	\includegraphics[width=\textwidth]{./Figures/System/%
	DecisionDirectedOperation.png}
	\caption{LMS in Decision Directed Operation}
	\label{fig:DDLMS}
\end{figure}
This region of operation introduces some delay to the system as %
the decision slicer needs to map the received symbol onto the %
nearest constellation point before the adaptive step can begin. %
As discussed in section \ref{sec:TIModel} the decision slicer %
being used here is MATLAB's \mintinline{MATLAB}{comm.QPSKDemodulator} %
from the Communications System Toolbox.

\section{USRP Implementation of the Time Varying System Model}

Figure \ref{fig:USRPDiagram} shows the hardware diagram %
for the radio front-end of the national instruments %
software defined radio. The radio front end uses a homodyne %
receiver architecture over a more traditional heterodyne %
architecture as shown in the DUC (Direct Up Conversion) and %
DDC (Direct Down Conversion) blocks in the diagram. The homodyne %
receiver architecture offers a simpler hardware receiver design %
which is in line with the software defined radio principle of %
simplifying electronics and making signal processing more %
reconfigurable \cite{Dillinger03}.
\begin{figure}[ht]
	\includegraphics[width=\textwidth]{./Figures/System/%
	USRP2943R.png}
	\caption{System Diagram for the Software Defined Radio%
	\cite{USRPDiagram}}
	\label{fig:USRPDiagram}
\end{figure}

\subsection{Direct Conversion Architecture}
\label{subsec:DirectConversion}
The direct conversion or homodyne architecture converts received %
bandpass signals directly to baseband signals without the need for %
an intermediate frequency. This is done by selecting a local %
oscillator (LO) frequency to be the same as the centre %
frequency of the bandpass signal.
\begin{figure}[ht]
	\centering
	\includegraphics[width=\textwidth]{./Figures/System/%
	DirectConversionScheme.png}
	\caption{Direct Down Conversion Process}
	\label{fig:DDC}
\end{figure}
Figure \ref{fig:DDC} illustrates this process where $I(t)$ is the %
in phase component and $Q(t)$ is the quadrature component. %
Although the direct conversion process offers advantages to the %
heterodyne architecture in terms of hardware simplicity, it comes %
with a key disadvantage in that if any part of the LO signal %
leaks into the RF input stream it introduces distortion into the %
received baseband signal as a DC offset. OFDM avoids %
additional interference caused by this LO leakage by inserting %
null carriers in and around the DC component.

\subsection{Transmitter Model}

The radio transmitter model was modeled on the finite %
synchronous transmit example model given in the LabView %
Instrument IO library. A diagram of transmitter is given %
in appendix \ref{app:TxSysModel}. %
The synchronous finite capture allows me to %
transmit and receive simultaneously, triggered on %
an internal clock, this removes the problem of %
synchronisation that would trouble a real %
wireless communication system and lets us %
constrain our attention to purely the effectiveness %
of the adaptive filtering algorithm at the receiver. %

In addition to the use of the synchronous transmit and %
receive an upsampling factor of $4$ has been used %
to ensure that the samples definitely occur on the %
modulated amplitude levels and not on any edges that %
occur to compensate for any fractional delay that %
might be present.


\subsection{Receiver Model}
\FloatBarrier
The radio receiver model was modeled on the finite %
synchronous receiver example model given in the Lab%
View Instrument IO library. A diagram of the receiver %
is given in appendix \ref{app:RxSysModel}. %

Unequalised the receiver constellation is a ring as is %
illustrated in figure \ref{fig:ReceivedRing}. The line %
leading from the centre of the ring to the edge is %
caused by the null carriers and the DC offset distortion %
caused by the LO as discussed in section \ref{subsec:Direct%
Conversion}. The ring itself is caused by the frequency offset %
between the LO that shifts the bandpass signal to the baseband %
and the central frequencies of each of the subcarriers. This %
frequency offset effect causes a rotation from the constellation %
point equal to the difference between the LO frequency and the %
central frequency of the subcarrier as described by 
\begin{align}
	r(t) = r'(t)e^{j 2\pi f_{e}t}
\end{align}
where $r'(t)$ is the ideal received constellation $r(t)$ is %
the received constellation and $f_{e}$ is the difference %
between the LO frequency and the subcarrier frequency.
\begin{figure}[ht]
	\centering
	\includegraphics[width=0.6\textwidth]{./Figures/System/%
	UnequalisedRing.png}
	\caption{Received Constellation before equalisation}
	\label{fig:ReceivedRing}
\end{figure}
\FloatBarrier
\subsection{Equalisation and Top Level Design}
The equalisation step in the USRP model follows %
the diagram in figure \ref{fig:USRPEqualisation}. The transmitted %
symbols are passed to the equalisation block and the received %
symbols are also passed to the equalisation block. Within this block %
the LMS, NLMS and zero forcing equalisers are implemented as %
described in chapter \ref{chap:AdaptiveFiltering} and %
\ref{subsec:Equalisation}.
\begin{figure}[ht]
	\centering
	\includegraphics[width=0.6\textwidth]{./Figures/System/%
	USRPEqualisation.png}
	\caption{USRP Equalisation}
	\label{fig:USRPEqualisation}
\end{figure}
\subsection{AWGN}
NI Labview does not provide a simple API for generating %
complex, circularly symmetric, gaussian noise. Instead, complex %
noise is generated directly from the definition of complex, circularly %
symmetric random variables. Such a variable $z \sim \mathcal{CN}(0,1)$ %
has independent and identically distributed (i.i.d) real and imaginary %
components, each distributed as $\sim \mathcal{N}(0,1/2)$. A series 5
of $n$ of these variables forms a standard circularly symmetric %
gaussian random vector \cite{Tse05}.

Instead, %
complex noise is generated given the properties that %
a complex, circularly symmetric random variable $z$ is %
distributed as $\sim\mathcal{CN}(0,1)$ has independent %
and identically distributed (i.i.d) real and imaginary %
components each distributed as $\sim\mathcal{N}(0,1/2)$, %
and that a collection of $n$ of these complex random %
variables forms a standard circularly symmetric %
gaussian random vector \cite{Tse05}.

Following from these properties random real and %
random noise components are generated independently %
with $\sigma_{\text{real}^{2}}$ and $\sigma_{\text{imag}^{2}}$ %
defined as half that of $\sigma_{n}^{2}$, where $\sigma_{n}^{2}$ %
is the power of the complex gaussian noise. The resulting %
real and imaginary random vectors are then summed
\begin{align}
	\mathbf{Z} = \mathbf{X} + i\mathbf{Y}
\end{align}
where $\mathbf{X}$ is the random gaussian vector %
representing the real component, and $\mathbf{Y}$ %
is the random gaussian vector representing the %
imaginary component.

\begin{figure}[ht]
	\includegraphics[width=\textwidth]{./Figures/%
	System/USRPNoiseGeneration.png}
	\caption{USRP noise generation}
	\label{fig:USRPNoiseGen}
\end{figure}

\subsection{Multipath Channel Simulation}
\label{sec:USRPMultipathChannelSimulation}
Similarly Labview does not provide an API for rayleigh %
fading channel models. The system model shown below %
takes the time-varying multipath coefficients output %
from MATLAB's \mintinline{MATLAB}{comm.RayleighChannel} and %
\mintinline{MATLAB}{comm.RicianChannel} and performs the discrete time %
convolution.

\begin{figure}[ht]
	\includegraphics[width=\textwidth]{./Figures/%
	System/USRPSimulatedChannel.png}
	\caption{USRP simulated channel model}
	\label{fig:USRPSimChan}
\end{figure}

This is performed before adding AWGN similarly to how the %
time varying and time invariant models. The addition of %
a real line of sight wireless channel in the radio introduces %
unknown attenuation and rotation to the simulated channel, %
this additional unknown introduces additional approximations %
that need to be made to evaluate the Wiener solution that %
were not needed in the time-varying MATLAB simulations, and %
were not evaluated mostly due to a lack of time in this project.

\begin{figure}[ht]
	\centering
	\includegraphics[width=0.5\textwidth]{./Figures/%
	System/USRP.png}
	\caption{USRP radio link at 2.45GHz}
	\label{fig:USRPChannel}
\end{figure}
