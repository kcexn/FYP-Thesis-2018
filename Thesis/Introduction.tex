\chapter{Introduction}
\label{ch:Introduction}
Wireless communications systems suffer from a wide %
range of impairments that make the communications %
channel a hostile environment to communicate over. %
The convenience of wireless communication over %
wired communication drives demand for accomodating %
the wireless environment. %
As the internet of things becomes more ubiquitous %
it will be necessary to develop methods of %
equalisation and filtering that are well suited to the %
wireless channel and are robust enough to function %
on cheaper hardware. Here, the motivating %
scenario for the study done throughout this %
report will be developed as well as a surface level %
coverage of the three major theoretical components, %
orthogonal frequency division multiplexing (OFDM) %
modulation, wireless channel modeling, and %
adaptive filtering. Chapters \ref{chap:OFDM}, %
\ref{chap:ChannelModeling}, and %
\ref{chap:AdaptiveFiltering} will give a more detailed %
coverage of the theory for OFDM, wireless channel %
modeling and adaptive filtering respectively. Chapters %
\ref{chap:System}, and \ref{chap:Results} will cover %
the experimental system model and the experiment %
results respectively. Chapter \ref{chap:Conclusion} %
draws some conclusions and provides some future %
directions of study.

\section{Motivating Scenario}
\label{sec:MotivatingScenario}
The next generation of wireless communication %
dubbed 5G makes promises of a densely connected %
network with numerous devices coming online % TODO: cite this
connected by the radio network \cite{Demestichas13,Hossain15}. %
With this kind of communication network in mind, it is easy %
to imagine a scenario in which a cell needs %
to serve many near static users a steady stream %
of data over a long period of time.% 
\begin{figure}[h!]
	\includegraphics[width=\linewidth]{./Figures/%
	Application_Scenario.png}
	\caption{Application Scenario}
	\label{fig:AppScene}
\end{figure}
Figure \ref{fig:AppScene} visualises the motivating %
scenario in mind of one base station serving many %
terminals over a bidirectional link. I'll be constraining %
the study done within this report to simply the down-link % 
between the base station and a terminal. This link will %
have a few properties of interest to us. The first is that %
the terminal will be nearly static with very little movement, %
the second, is that the environment is prone to change. We can %
imagine this scenario as being a user, sitting down in a busy %
courtyard watching streaming video on their mobile phone. The %
user equipment is unlikely to move much however the environment %
connecting the user to the base station is prone to change as %
people and vehicles move about. I will model this environment %
as one that will introduce slow time-variations to the wireless %
channel connecting the station to the user. Although this %
particular example is an oversimiplification of how streaming %
video works over standard communication protocols, the idea %
of internet of things (IoT) devices needing to operate in %
slowly time varying conditions efficiently is not too far %
of a stretch from this simple example.

\section{Channel Modeling}

There are three major types of impairments that affect %
wireless communication.
\begin{itemize}
	\item{Path loss}
	\item{Thermal electronic noise}
	\item{Fading}
\end{itemize}
Path loss is an impairment that affects %
all forms of electronic communication and is %
caused by the natural decay in amplitude %
that electromagnetic waves undergo as they %
propagate through free space. Path loss %
is proportional to $\frac{1}{d^2}$ where %
$d$ is the distance between the transmitter %
and receiver.

Thermal electronic noise is the noise %
caused by the electronics at the %
transmitter and receiver. The measure %
of thermal noise that is important %
will be the ratio between average %
signal power and average noise power %
commonly referred to as signal-to-%
noise ratio (SNR). The noise is %
commonly fixed by the temperature %
of operation for the electronics. %
The noise power spectral density %
is defined as %
\begin{align}
	N_{0} = \kappa T
\end{align}
where $\kappa$ is the boltzmann %
constant and $T$ is the temperature %
of operation in Kelvin. Noise %
power is typically relative constant %
in a given system and is defined as
\begin{align}
	\sigma^{2} = BT
\end{align}
where $B$ is the bandwidth of the %
signal of interest.

Fading is a stochastic impairment %
that is unique to wireless %
channels and is caused by obstructions %
in the surrounding environment, %
relative movement between the %
transmitter and receiver %
or movement of objects between %
the transmitter and receiver. The %
combination of movement and %
obstructions cause the wave to %
reflect and scatter throughout %
throughout the environment of %
interest before arriving %
at the receiver. Fading is %
characterised by four functions, %
coherence time, which is a measure %
of how long the fading channel %
remains constant, coherence bandwidth, %
a measure of how correlated two neighbouring %
frequencies are with respect to frequency %
separation, doppler spread, a measure of %
the frequency spread caused by doppler shift, %
and the power delay profile, a measure of %
the amount of pulse spreading in time. %
These functions define two main properties %
of fading, the rate of fading (slow or fast), %
and the frequency selectivity of the fading %
(how distorted the signal of interest is %
in the frequency domain). A deeper discussion %
of these concepts is discussed in chapter %
\ref{chap:ChannelModeling} where a description %
of the channel as finite impulse response filter %
is also provided.

\section{Orthogonal Frequency Division Multiplexing %
(OFDM)}
\label{sec:OFDMIntro}
The simplest way to modulate signals over the wireless %
channel is to use some form of single carrier modulation. %
Single carrier modulation takes an input data stream, %
separates the stream into in-phase and quadrature components %
if necessary, pulse shapes, the signal, up converts to the %
carrier frequency and transmits over the wireless channel. %
There are many methodologies of single carrier modulation %
however in all methods of single carrier modulation %
the symbol time is inversely proportional the transmission %
bandwidth, $T_{s} \propto 1/B$. This means that in order to %
achieve faster data rates, more bandwidth is always required. %
In frequency selective wireless channels, this induces intersymbol %
interference (ISI) that can extend for many symbols. This ISI %
can introduce a complex and computationally expensive equalisation %
operation to undo the distortion introduced.

As will be discussed in more detail in chapter %
\ref{chap:ChannelModeling} the wireless propagation medium can be %
modeled as a finite impulse response filter\cite{Jer00}. 
As the filter length increases, estimating %
the channel becomes an increasingly more complex task as %
equalisation complexity grows with filter length. %
Orthogonal frequency division multiplexing or OFDM significantly %
helps to simplify the problem of equalisation caused by %
the frequency selectivity of the channel. OFDM works by %
separating the bandwidth of the signal of interest into %
$N$ subcarriers each with a subband bandwidth of %
$B_{\text{total}}/N$. A key advantage of these subcarriers %
is that the bandwidths of each of the OFDM subcarreirs can %
be designed to be narrower than the coherence bandwidth %
of the frequency selective wireless channel. As will be %
developed in chapter \ref{chap:AdaptiveFiltering} each %
subcarrier can be equalised independently with a single %
coefficient that approximates the locally flat properties %
of the fading channel. Chapter \ref{chap:OFDM} develops %
the mathematical construction of OFDM and develops %
properties of orthogonality and circular convolution.
\newpage
\section{Adaptive Filtering}

In section \ref{sec:OFDMIntro} OFDM was introduced %
as a solution to the equalisation complexity introduced %
by the frequency selectivity of the wireless channel. In %
the time varying scenario described in section %
\ref{sec:MotivatingScenario}, a time invariant %
estimate of the wireless channel will need %
to be periodically updated with time in order %
to provide adequate filtering.

Adaptive filters offer a method to track changes in the %
wireless channel over time, so that the equalisation filter %
can follow the optimal solution over time. The filters %
studied in this report will be the least mean square (LMS) %
filter and the normalised least mean square (NLMS) filter. Both %
filters are considered members of the class of filters known %
as stochastic gradient filters. Stochastic gradient filters %
minimise mean square error between the received symbol %
and a desired output by finding the stationary point %
of the paraboloid that defines the mean square error.
\begin{figure}[ht]
	\centering
	\includegraphics[width=0.4\textwidth]{./Figures/%
Circular_Paraboloid_Quadric.png}
	\caption{Circular parabaloid in three dimensions 
	\cite{Paraboloid12}}
	\label{fig:Paraboloid}
\end{figure}
A more formal definition of the method of gradient descent, %
mean square error, least mean square, and normalised least mean %
square will be provided in chapter \ref{chap:AdaptiveFiltering}
