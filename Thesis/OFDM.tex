\chapter{Orthogonal Frequency Division Multiplexing (OFDM)}

Chapter 1 introduced the idea of OFDM and why it's %
advantageous to use. This chapter will aim to develop %
OFDM mathematically and provide some theoretical insight %
into how and why it works.
\begin{figure}[h!]
	\begin{subfigure}{\linewidth}
		\includegraphics[width=\linewidth]{./Figures/Chapter2/%
		OFDM_Transmit.png}
		\caption{OFDM Transmitter block diagram}	
		\label{subfig:OFDM_Transmitter}
	\end{subfigure}
	\begin{subfigure}{\linewidth}
		\includegraphics[width=\linewidth]{./Figures/Chapter2/%
		OFDM_Transmit.png}
		\caption{OFDM Receiver block diagram}
		\label{subfig:OFDM_Receiver}
	\end{subfigure}
	\caption{OFDM transmission and reception diagrams}
	\label{fig:OFDM}
\end{figure}

Figure \ref{fig:OFDM}\subref{subfig:OFDM_Transmitter}. %
depicts the block diagram of the OFDM transmitter.

\section{Discrete Fourier Transform and Fast Fourier Transform}

It's long been known that signals can be overlapped in the %
frequency domain with orthogonal frequency spacing %
allowing for efficient utilisation of the %
frequency spectrum\cite{Chang66}.

A key driver to the efficiency and widespread adoption %
of OFDM is the discrete fourier transform (DFT) defined %
as\cite{Rao2010}:

\begin{align}
	X^F(k) = \sum_{n=0}^{N-1}x(n)e^{(\frac{-i 2\pi}{N})kn}, %
	k=0,1,\cdots,N-1,\quad N \text{ DFT coefficients}
\end{align}

And the inverse transform is defined as:

\begin{align}
	x(n) = \frac{1}{N}\sum_{k=0}^{N-1}X^{F}(k)%
	e^{(\frac{i 2\pi}{N})kn}, n = 0,1,\cdots,N-1, \quad%
	N \text{ data samples}
	\label{eq:IDFT}
\end{align}

The discrete fourier transform has the orthogonality %
properties necessary for orthogonal spectral allocation. %
To show this I'll follow the development in \cite{Opp99}. %
multiply both sides of eq \ref{eq:IDFT}. by %
$e^{-j(\frac{2\pi}{N})rn}$ and summing from $n=0$ %
to $n=N-1$ gives:

\begin{align}
	\sum_{n=0}^{N-1}x(n)e^{(\frac{-i 2\pi}{N})rn} = %
	\sum_{k=0}^{N-1}X^{F}(k)\left[\frac{1}{N} \sum_{n=0}^{N-1} %
	e^{(\frac{i 2\pi}{N})(k-r)n}\right]
\end{align}

evaluating the exponential:

\begin{align}
	\frac{1}{N}\sum_{n=0}^{N-1}e^{(\frac{i 2\pi}{N})(k-r)n}%
	= \begin{cases}
		1,\quad k-r = mN, \quad m\text{ an integer},\\
		0,\quad \text{otherwise.}
	\end{cases}
\end{align}

which expresses the orthogonality of the complex exponentials %
$e^{(\frac{i 2\pi}{N})kn}$ and $e^{(\frac{-i 2\pi}{N})rn}$ for %
integer values of $k$ and $r$. This can be interpreted as the %
complex exponentials forming an orthonormal basis proving that %
the DFT and IDFT have the orthogonality properties needed for %
OFDM.

By itself the DFT has computational complexity of $\mathcal{O}(%
N^2)$ as there are $N$ multiplication for each coefficient and %
$N$ coefficients in the DFT. Fast versions of the DFT which %
reduce the computational complexity of the DFT to $\mathcal{O}(%
Nlog(N))$ have been fundamental to the adoption of OFDM.%

The fast fourier transform (FFT) used in the remainder of %
this report % TODO: thesis? 
is the default FFT provided by MATLAB built on FFTW\cite{FFTW}. %
\section{Cyclic Prefix}

%TODO: develop the mathematical reasoning for the cyclic
% prefix
