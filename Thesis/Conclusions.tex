\chapter{Conclusions and Future Directions}

\section{Concluding Remarks}

A variety of adaptive filtering algorithms have been tested %
and their performance in the learning and decision directed %
schemes measured. Several things can be concluded from the %
results provided in chapter \ref{chap:Results}, firstly the %
frequency domain adaptive filtering model described in section %
\ref{sec:TIModel} performs as expected. The single tap %
variant of the LMS and NLMS algorithm converges to the %
Wiener solution as expected and exhibits excess mean square error %
that increases with increasing step size. 

Section \ref{sec:TVResults} %
results reveal some interesting properties about the LMS and %
NLMS algorithms in both convergence to the wiener solution and tracking %
of the time varying channel in a decision directed manner. three different %
parameters were varied to see what effect they had on the performance %
of the LMS and NLMS filters in both the learning and decision directed modes, %
the length of the cyclic prefix which will make a difference on how much change %
there is between OFDM symbols, the coherence time, and the SNR. In %
decision directed operation the LMS algorithm overall performs better than %
the NLMS algorithm in tracking the change over time regardless of whether %
the SNR is high or low. The results indicate that the LMS algorithm performs better %
at lower SNR's than high ones, this is likely due to the fact that in a purely %
learning state at high SNR's, the LMS algorithm is unable to converge to the %
wiener solution, this is particularly visible in figure \ref{fig:LMS-Short-High-None} where %
the LMS solution converges to a result that is significantly higher than the %
Wiener solution, the LMS's fixed step size and relatively slow rate of convergence %
is likely to be the culprit in preventing the LMS to converge to the optimal %
solution under these parameters. At lower SNR's the optimal solution has a %
much higher mean square error and so the LMS algorithm even with the slower %
rate of convergence is able to achieve the Wiener solution. %
In contrast to this the NLMS algorithm is always capable of converging to the %
Wiener solution in both high and low SNR's, this is distinctly better than %
the performance exhibited by the LMS algorithm. A peculiar result occurs at %
low SNR's for the NLMS algorithm in the sense that it appears to overfit the %
optimal solution by a significant margin as can be seen in figure %
\ref{fig:NLMS-Long-Low-None-Long}, it's unclear why this occurs. % TODO: mention this in future directions

The decision directed performance of the NLMS in both low and high %
SNR's is considerably worse than that of the LMS algorithm, this is %
attributable to the fact that the NLMS algorithm takes significantly %
larger steps than the LMS algorithm towards what it believes is %
an optimal solution, so when an incorrect decision is made by the %
decision slicer the filter coefficients may change significantly, %
a property that makes this problem particularly worse is that unlike %
the LMS algorithm's performance in figure \ref{fig:Medium-High-Directed-Long} 
the NLMS filter coefficients may never recover even at high SNR's as %
is clear in figure \ref{fig:NLMS-Short-High-Directed-Short}.

The results from section \ref{sec:USRPResults} demonstrate that %
the mean square error convergence on the software defined %
radio does converge in a way that is consistent with the MATLAB %
simulations. It was not possible to examine the decision directed %
performance of the radio implementation with any kind of rigour %
as the unknown channel between the antennae meant that %
a wiener solution cannot be found and as was made clear in section %
\ref{sec:TVResults} the LMS algorithm does not guarantee %
convergence to the optimal solution under a time varying scenario, and %
the NLMS algorithm has a tendency to overfit. Comparing figures %
\ref{fig:NLMS-BER-USRP} with \ref{fig:NLMS-BER-Long-DD-TV}, it %
clear that this overfitting behaviour is present in the radio implementation %
as well.

Overall the length of the cyclic prefix seemed to have minimal effect %
on the system performance so long as it is longer than the %
channel memory so that the circular convolution described in %
chapter \ref{chap:OFDM} still applies.

A good frequency domain adaptive receiver should require a short %
number of training symbols to initialise itself to the Wiener solution %
so that most of the symbols transmitted can be useful data. These %
results indicate that the NLMS algorithms significantly faster convergence %
time in training make it a better choice for initialisation. The NLMS algorithms %
poor performance in decision directed operation make the standard LMS %
algorithm a better choice for tracking the time-varying nature of the channel %
without additional training symbols. Close attention should be paid to the %
coherence time, coherence bandwidth, and SNR as these channel parameters %
will greatly influence choices of OFDM FFT length, OFDM symbol time and %
step size.

\section{Future Directions}

There are many improvements and areas of study left to %
investigate in this system. Firstly, it is clear that faster %
converging algorithms such as the NLMS are significantly %
better for initialisation to the Wiener solution than slower %
converging algorithms. References \cite{Qureshi77} and %
\cite{Crozier91} suggest that some training sequences %
may have properties that allow for fast convergence %
to the Wiener solution, and so more intelligent choices of %
training sequence is worth investigating. Other types of %
adaptive filters may also be worth looking at such %
as the recursive least square (RLS), given that %
OFDM reduces the channel estimation problem to that %
of a single filter coefficient, it may be computationally %
tractable to look at more computationally complex filters %
such as an affine projection filter or variants on the Kalman %
filter. 

Reference \cite{Wei17} suggests that both time and frequency %
coherence can be exploited for equalisation. Investigating %
the performance of introducing weighted filter coefficient %
interpolation in the frequency domain may significantly %
improve the performance of the adaptive receiver in decision %
directed operation. In addition if the continuous stream of data %
is long, periodic retraining symbols can be transmitted to %
provide corrections to accumulated errors under decision %
directed operation, in combination with frequency domain %
coefficient interpolation, it may not be necessary to %
retransmit these retraining symbols on all the subcarriers. If %
quality of service information can be transmitted upstream %
to the base station and can be responded to in real time, it %
may also be feasible to only selectively resend training %
symbols on subcarriers that are experiencing significant %
departure from the optimal solution. 

In section \ref{sec:TVResults} it was made apparent that the %
NLMS algorithm performed poorly in decision directed operation %
due to the fact that it steps further away from the correct %
decision when the decision slicer makes an incorrect decision. %
The number of incorrect decisions made due to noise, the density %
of the symbol constellation (i.e., 16-QAM, 64-QAM etc.) will both %
have a significant impact on how well these adaptive filters will %
perform while decision directed. This study does not spend any %
time evaluating the relationship between constellation density, %
noise and decision directed performance, and it may be %
important to characterise this as higher constellation density %
may reduce decision directed performance to an impractical %
level at similar $E_b/N_0$, so additional $E_b/N_0$ may be %
required.

% TODO: Develop future study directions for this.
