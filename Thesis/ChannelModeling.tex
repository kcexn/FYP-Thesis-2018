\chapter{Wireless Channel Modeling}

The digital communication channel as described in %
% TODO: find the textbook I think by Sklar that 
% describes the system model. 
[Sklar] %TODO: replace this.

Fading is a major component of wireless channel %
impairments. Fading comes under two main categories %
large scale and small scale fading.

% TODO: give the text book description of large scale fading

\section{Fading Channels}

\subsection{Large Scale Fading}
Large scale %
fading is characterised by effects such as shadowing %
where large obstructions within the environment such as %
buildings significantly attenuate the wireless signal %
before reaching the receiver. This type of fading is %
traditionally modeled using a log-normal distribution %
as has been experimentally shown by 
% Cite Hata and his crew.

This kind of fading is known to be at least partially %
compensated by power control\cite{Jer00}. In addition, %
referring back to my motivating scenario in section %
\ref{sec:MotivatingScenario}, the relatively stationary %
environment that I'm operating within means that any time-varying %
effects that might be introduced by large-scale fading %
will be very small relative to the small scale fading. 

\subsection{Small Scale Fading}
% TODO: give the textbook description of small scale fading

Small scale fading is caused by a transmitted signal %
arriving at the receiver with slightly different delays %
and angles after having been scattered or reflected by %
the environment in some way. The signals all self %
interfere at the receiver causing constructive or %
destructive interference. 

% TODO: make a diagram depicting the self interference

Small scale fading is typically modeled as having an %
envelope that follows a Rayleigh distribution or a %
Rician distribution, normally referred to as Rayleigh %
fading and Rician fading respectively. % TODO: maybe reference
% simulation of communication systems

% TODO: make a diagram of the Rayleigh distribution
% TODO: make a diagram of the Rician distribution
% TODO: Enter the expressions for a Rayleigh distribution
% TODO: Enter the expressions for a Rician distribution
% TODO: find and plot a measured Rayleigh distribution

Good modeling and simulation of this small scale fading %
is crucial to the accurate simulation of wireless channels. %
There are two main methods of simulating small scale fading, %
the sums of sinusoids method first developed by Jakes in %
\cite{Jakes74} and a filtered white gaussian noise method. %
The filtered white gaussian noise methodology is the chosen %
method of simulation for this report.

\subsection{Wireless Channel as a Filter}

% TODO: Develop Finite impulse response filter model 
% of the wireless channel

% TODO: develop the model for the time-varying characteristics
% of wireless channels

% TODO: Develop the mathematics of the finite gaussian channel model


